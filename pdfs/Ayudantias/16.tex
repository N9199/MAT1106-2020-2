\documentclass{ayudantia}
\usepackage{multicol}
\title{Ayudantía 16}
\date{2020-10-27}
\course{MAT1106 --- Introducción al Cálculo}

% Comment for final compile
\ifx\condition\undefined
\def\condition{1}
\fi

\ifx\condition\undefined
\immediate\write18{ pdflatex -synctex=1 -output-directory="../Enunciados" --jobname="Enunciado\jobname" "\gdef\string\condition{0} \string\input\space\jobname"} 
\immediate\write18{ pdflatex -synctex=1 -output-directory="../Soluciones" --jobname="Solucion\jobname" "\gdef\string\condition{1} \string\input\space\jobname"} 

\immediate\write18{ cd "../Enunciados" && rm *.aux *.log *.out}
\immediate\write18{ cd "../Soluciones" && rm *.aux *.log *.out}

\expandafter\stop
\fi

\ifcase\condition
\excludecomment{ans}
\or
\includecomment{ans}
\fi

\begin{document}
\maketitle



\begin{prob}
    Sea \(x_n=\frac1{n^k}\binom{n}{m}\) con \(k,m\in\set{N}\):
    \begin{enumerate}[label=(\alph*)]
        \item Demuestre que si \(k<m\) se tiene que \(x_n\rightarrow\infty\)
        \item Demuestre que si \(k=m\) se tiene que \(x_n\rightarrow\frac1{m!}\)
        \item Demuestre que si \(k>m\) se tiene que \(x_n\rightarrow0\)
    \end{enumerate}
\end{prob}

\begin{ans}
    \begin{sol}

    \end{sol}
\end{ans}



\begin{prob}
    Sea \(x_n\) una sucesión convergente y \(\varepsilon>0\), demuestre que existe una subsucesión \(x_{n_k}\) tal que para todo \(k\in\set{N}\) se tiene
    \begin{equation*}
        \abs{x_{n_k}-x_{n_{k+1}}}<\varepsilon.
    \end{equation*}
\end{prob}

\begin{ans}
    \begin{sol}

    \end{sol}
\end{ans}

\begin{prob}
    Sea \(x_n\) una sucesión. Definimos \(s_n=\sum_{k=1}^nx_k\). Asuma que \(s_n\rightarrow L\) y que \(x_n\) es siempre positiva. Definimos
    \begin{equation*}
        r_n=\lim_{m\rightarrow\infty}\sum_{k=n+1}^mx_k.
    \end{equation*}
    \begin{enumerate}[label=(\alph*)]
        \item Encuentre \(r_n\) de manera explicita.
        \item Demuestre que \(r_n\rightarrow 0\).
    \end{enumerate}
\end{prob}

\begin{ans}
    \begin{sol}

    \end{sol}
\end{ans}

\end{document}





