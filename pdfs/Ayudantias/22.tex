\documentclass{ayudantia}
\usepackage{multicol}
\title{Ayudantía 22}
\date{2020-11-17}
\course{MAT1106 --- Introducción al Cálculo}

% Comment for final compile
\ifx\condition\undefined
\def\condition{1}
\fi

\ifx\condition\undefined
\immediate\write18{ pdflatex -synctex=1 -output-directory="../Enunciados" --jobname="Enunciado\jobname" "\gdef\string\condition{0} \string\input\space\jobname"} 
\immediate\write18{ pdflatex -synctex=1 -output-directory="../Soluciones" --jobname="Solucion\jobname" "\gdef\string\condition{1} \string\input\space\jobname"} 

\immediate\write18{ cd "../Enunciados" && rm *.aux *.log *.out}
\immediate\write18{ cd "../Soluciones" && rm *.aux *.log *.out}

\expandafter\stop
\fi

\ifcase\condition
\excludecomment{ans}
\or
\includecomment{ans}
\fi

\begin{document}
\maketitle

\begin{prob}
    Demuestre que los siguientes son equivalentes:
    \begin{itemize}
        \item \(s\) es el supremo de \(A\).
        \item Para todo \(\varepsilon>0\) existe un \(a\in A\) tal que \(s-\varepsilon< a\leq s\).
        \item \(s\) es cota superior de \(A\) y existe una sucesión \(\{x_n\}_{n\in\set{N}}\) de elementos de \(A\) tal que \(x_n\rightarrow s\).
    \end{itemize}
\end{prob}

\begin{ans}
    \begin{sol}

    \end{sol}
\end{ans}



\begin{prob}
    Sean \(A,B\) dos conjuntos de números reales no vacíos, se definen
    \begin{equation*}
        A+B=\{a+b\mid a\in A\wedge b\in B\}
    \end{equation*}
    Demuestre que \(A+B\) tiene máximo si y solo si \(A\) tiene máximo y \(B\) tiene máximo.
\end{prob}

\begin{ans}
    \begin{sol}

    \end{sol}
\end{ans}



\begin{prob}
    Sea \(A\neq\emptyset\) un conjunto de números reales, y sea \(x_n\) una sucesión de cotas superiores de \(A\) que converge a \(L\in A\).
    \begin{enumerate}
        \item Demuestre que \(L\) es una cota superior.
        \item Demuestre que \(L\) es el supremo de \(A\).
    \end{enumerate}
\end{prob}

\begin{ans}
    \begin{sol}

    \end{sol}
\end{ans}



\begin{prob}
    Sea \(f:\set{R}\rightarrow\set{R}\) una función, y sea \(X\) un conjunto no vacío y acotado. ¿Se puede concluir que \(f(X)\) tiene supremo o ínfimo?
\end{prob}

\begin{ans}
    \begin{sol}

    \end{sol}
\end{ans}
\end{document}










