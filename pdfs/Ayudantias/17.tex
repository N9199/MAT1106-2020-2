\documentclass{ayudantia}
\usepackage{multicol}
\title{Ayudantía 17}
\date{2020-10-29}
\course{MAT1106 --- Introducción al Cálculo}

% Comment for final compile
%\ifx\condition\undefined
%\def\condition{1}
%\fi

\ifx\condition\undefined
\immediate\write18{ pdflatex -synctex=1 -output-directory="../Enunciados" --jobname="Enunciado\jobname" "\gdef\string\condition{0} \string\input\space\jobname"} 
\immediate\write18{ pdflatex -synctex=1 -output-directory="../Soluciones" --jobname="Solucion\jobname" "\gdef\string\condition{1} \string\input\space\jobname"} 

\immediate\write18{ cd "../Enunciados" && rm *.aux *.log *.out}
\immediate\write18{ cd "../Soluciones" && rm *.aux *.log *.out}

\expandafter\stop
\fi

\ifcase\condition
\excludecomment{ans}
\or
\includecomment{ans}
\fi

\begin{document}
\maketitle

\begin{prob}
    Sea \(x_n\) una sucesión de términos no cero tal que \(x_n\rightarrow 0\). Sea
    \begin{equation*}
        \lambda_n=\frac{(1+x_n)^k-1}{x_n}
    \end{equation*}
    con \(k\in\set{N}\) fijo. Encuentre \(\lim_{n\rightarrow\infty}\lambda_n\).
\end{prob}

\begin{ans}
    \begin{sol}
        
    \end{sol}
\end{ans}



\begin{prob}
    Sean \(a_1,\ldots,a_k\in\set{R}\).
    \begin{enumerate}[label=(\alph*)]
        \item Demuestre que \(\sqrt[n]{\abs{a_1}^n+\ldots+\abs{a_k}^n}\leq\max\{\abs{a_1},\ldots,\abs{a_k}\}\).
        \item Demuestre que \(\lim_{n\rightarrow\infty}\sqrt[n]{\abs{a_1}^n+\ldots+\abs{a_k}^n}=\max\{\abs{a_1},\ldots,\abs{a_k}\}\)
    \end{enumerate}
\end{prob}

\begin{ans}
    \begin{sol}

    \end{sol}
\end{ans}



\begin{prob}
    Considere \(I_n=[a_n,b_n]\), donde \(a_n\) es creciente, \(b_n\) es decreciente y \(a_n\leq b_n\) para todo \(n\). Demuestre que \(\bigcap_{n\in\set{N}}I_n\neq\emptyset\). ¿Qué pasaría si los intervalos fueron abiertos?
\end{prob}

\begin{ans}
    \begin{sol}

    \end{sol}
\end{ans}



\begin{prob}
    Sea la sucesión
    \begin{equation*}
        \sqrt{k},\sqrt{k+\sqrt{k}},\sqrt{k+\sqrt{k+\sqrt{k}}},\ldots
    \end{equation*}
    con \(k\in\set{N}\).
    \begin{enumerate}[label=(\alph*)]
        \item Demuestre que si \(k=2\), la sucesión converge.
        \item Demuestre que la sucesión está acotada para cualquier \(k\in\set{N}\) fijo.
        \item Encuentre condiciones necesarias y suficientes para que la sucesión converja a un número entero.
    \end{enumerate}
\end{prob}

\begin{ans}
    \begin{sol}

    \end{sol}
\end{ans}

\end{document}






