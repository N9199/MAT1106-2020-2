\documentclass{ayudantia}
\usepackage{multicol}
\title{Ayudantía 06}
\date{2020-09-03}
\course{MAT1106 --- Introducción al Cálculo}

% Comment for final compile
%\ifx\condition\undefined
%\def\condition{1}
%\fi

\ifx\condition\undefined
\immediate\write18{ pdflatex -synctex=1 -output-directory="../Enunciados" --jobname="Enunciado\jobname" "\gdef\string\condition{0} \string\input\space\jobname"} 
\immediate\write18{ pdflatex -synctex=1 -output-directory="../Soluciones" --jobname="Solucion\jobname" "\gdef\string\condition{1} \string\input\space\jobname"} 

\immediate\write18{ cd "../Enunciados" && rm *.aux *.log *.out}
\immediate\write18{ cd "../Soluciones" && rm *.aux *.log *.out}

\expandafter\stop
\fi

\ifcase\condition
\excludecomment{ans}
\or
\includecomment{ans}
\fi

\begin{document}
\maketitle

\begin{prob}
    Encuentre el conjunto de solución de la inecuación
    \begin{equation*}
        \frac3{1-x}<\frac{x+6}{2-x}
    \end{equation*}
\end{prob}

\begin{ans}
    \begin{sol}
        Se nota que \(x\neq 2\) y \(x\neq 1\), y se ve lo siguiente:
        \begin{align*}
            \frac3{1-x}<\frac{x+6}{2-x} & \iff 0<\frac{x+6}{2-x}-\frac3{1-x}          \\
                                        & \iff 0<\frac{(x+6)(1-x)-3(2-x)}{(2-x)(1-x)} \\
                                        & \iff 0<\frac{x+6-x^2-6x-6+3x}{(2-x)(1-x)}   \\
                                        & \iff 0< -\frac{x(x+2)}{(2-x)(1-x)}          \\
                                        & \iff 0< -\frac{x(x+2)}{(x-2)(x-1)}          \\
        \end{align*}
        Por lo que viendo la siguiente tabla:
        \begin{center}
            \begin{tabular}{c|c|c|c|c|c|}
                                               & \((-\infty,-2)\) & \((-2,0)\) & \((0,1)\) & \((1,2)\) & \((2,\infty)\) \\
                \hline\hline
                \(x+2\)                        &-&+&+&+&+\\
                \(x\)                          &-&-&+&+&+\\
                \(x-1\)                        &-&-&-&+&+\\
                \(x-2\)                        &-&-&-&-&+\\
                \hline
                \(-\frac{x(x+2)}{(x-2)(x-1)}\) &-&+&-&+&-\\
            \end{tabular}
        \end{center}
        Por lo que se tiene que \(x\in(-2,0)\cup(1,2)\).
    \end{sol}
\end{ans}


\begin{prob}
    Sea \(\alpha>0\). Encuentre todos los valores de \(x\) tales que
    \begin{equation*}
        \abs{x^2-\alpha^2}>\abs{x-\alpha}
    \end{equation*}
\end{prob}

\begin{ans}
    \begin{sol}
        Se nota que \(x\neq\alpha\), ya que si \(x=\alpha\) se tiene \(0\abs{x^2-\alpha^2}>\abs{x-\alpha}=0\). Y si \(x\neq\alpha\) se tiene que \(\abs{x-\alpha}>0\), por lo que \(\abs{x+\alpha}>1\). Por lo que \(x>1-\alpha\) o \(x<-1-\alpha\) por lo que \(x\in((-\infty,-1-\alpha)\cup(1-\alpha,\infty))\setminus\{\alpha\}\).
    \end{sol}
\end{ans}



\begin{prob}
    \textit{(I3 2017)} Sea \(z>0\) fijo, y sea \(A_z\) el conjunto de solución de la inecuación
    \begin{equation*}
        \abs{x^2+xz+z^2}\leq zx+2z^2.
    \end{equation*}
    Demuestre que si \(0<z_1<z_2\), entonces \(A_{z_1}\subseteq A_{z_2}\)
\end{prob}

\begin{ans}
    \begin{sol}
        Sabemos que \(x^2+zx+z^2\geq0\) por ayudantía pasada. Por lo que la inecuación es equivalente a
        \begin{equation*}
            x^2\leq z^2
        \end{equation*}
        Se sabe que el conjunto solución de esa inecuación es \([-z,z]\).

        Si \(0<z_1<z_2\), entonces \(-z_1>-z_2\). Sea \(x\in A_{z_1}\), esto nos dice que \(-z_1\leq x\leq z_1\). Por lo que \(-z_2< -z_1\leq x\leq z_1< z_2\) nos dice que \(x\in A_{z_2}\).
    \end{sol}
\end{ans}


\begin{prob}
    Demuestre la desigualdad de Nesbitt: Si \(a,b,c>0\) se tiene que
    \begin{equation*}
        \frac{a}{b+c}+\frac{b}{a+c}+\frac{c}{a+b}\geq\frac32
    \end{equation*}
\end{prob}

\begin{ans}
    \begin{sol}
        S.p.d.g. \(a\geq b\geq c\), por lo que
        \begin{equation*}
            \frac1{b+c}\geq\frac1{a+c}\geq\frac1{a+b}
        \end{equation*}
        Usando la desigualdad demostrada la ayudantía pasada se tiene que
        \begin{align*}
            \frac{a}{b+c}+\frac{b}{a+c}+\frac{c}{a+b} & \geq\frac{b}{b+c}+\frac{c}{a+c}+\frac{a}{a+b} \\
            \frac{a}{b+c}+\frac{b}{a+c}+\frac{c}{a+b} & \geq\frac{c}{b+c}+\frac{a}{a+c}+\frac{b}{a+b} \\
        \end{align*}
        Sumando ambas desigualdades se tiene
        \begin{equation*}
            2\paren{\frac{a}{b+c}+\frac{b}{a+c}+\frac{c}{a+b}}\geq3
        \end{equation*}
        Lo que es equivalente lo pedido.s
    \end{sol}
\end{ans}


\end{document}