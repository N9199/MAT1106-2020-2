\documentclass{ayudantia}
\usepackage{multicol}
\title{Ayudantía 11}
\date{2020-10-05}
\course{MAT1106 --- Introducción al Cálculo}

% Comment for final compile
%\ifx\condition\undefined
%\def\condition{1}
%\fi

\ifx\condition\undefined
\immediate\write18{ pdflatex -synctex=1 -output-directory="../Enunciados" --jobname="Enunciado\jobname" "\gdef\string\condition{0} \string\input\space\jobname"} 
\immediate\write18{ pdflatex -synctex=1 -output-directory="../Soluciones" --jobname="Solucion\jobname" "\gdef\string\condition{1} \string\input\space\jobname"} 

\immediate\write18{ cd "../Enunciados" && rm *.aux *.log *.out}
\immediate\write18{ cd "../Soluciones" && rm *.aux *.log *.out}

\expandafter\stop
\fi

\ifcase\condition
\excludecomment{ans}
\or
\includecomment{ans}
\fi

\begin{document}
\maketitle

\begin{prob}
    Sean \(x_n\) e \(y_n\) dos sucesiones tales que \(x_n\leq y_n\) para todo \(n\in\set{N}\). Demuestre que si \(x_n\) no es acotada superiormente entonces \(y_n\) no es acotada superiormente.
\end{prob}

\begin{ans}
    \begin{sol}

    \end{sol}
\end{ans}


\begin{prob}
    Demuestre que para todo par de números reales \(x,y\) distintos existe un racional \(z\) tal que \(x<z<y\).
    \textit{Hint: Usar propiedad arquimediana y parte entera.}
\end{prob}

\begin{ans}
    \begin{sol}

    \end{sol}
\end{ans}


\begin{prob}
    Demuestre que la sucesión
    \begin{equation*}
        x_n=\sum_{k=1}^n\frac1n
    \end{equation*}
    cumple que \(x_{2^n}\geq\frac{n+1}2\) para todo \(n\in\set{N}\).
\end{prob}

\begin{ans}
    \begin{sol}

    \end{sol}
\end{ans}


\begin{prob}
    Demuestre que todo sucesión creciente y no acotada \(x_n\) cumple que su límite existe y \(\lim_{n\rightarrow\infty}x_n=\infty\).
\end{prob}

\begin{ans}
    \begin{sol}

    \end{sol}
\end{ans}



\begin{prob}
    Sea \(x_n\) una sucesión se denota \(s_n\) a la sucesión de las sumas parciales:
    \begin{equation*}
        s_n=\sum_{k\leq n}x_k
    \end{equation*}
    Demuestre que si todos los términos de \(x_n\) son positivos, entonces \(s_n\) es creciente. Demuestre también que si para todo \(n\in\set{N}\) se tiene que \(x_n>\varepsilon\) para algún \(\varepsilon>0\), entonces \(\lim_{n\rightarrow\infty}s_n=\infty\).
\end{prob}

\begin{ans}
    \begin{sol}

    \end{sol}
\end{ans}

\end{document}
