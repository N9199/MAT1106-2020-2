\documentclass{ayudantia}
\usepackage{multicol}
\title{Ayudantía 11}
\date{2020-10-05}
\course{MAT1106 --- Introducción al Cálculo}

% Comment for final compile
%\ifx\condition\undefined
%\def\condition{1}
%\fi

\ifx\condition\undefined
\immediate\write18{ pdflatex -synctex=1 -output-directory="../Enunciados" --jobname="Enunciado\jobname" "\gdef\string\condition{0} \string\input\space\jobname"} 
\immediate\write18{ pdflatex -synctex=1 -output-directory="../Soluciones" --jobname="Solucion\jobname" "\gdef\string\condition{1} \string\input\space\jobname"} 

\immediate\write18{ cd "../Enunciados" && rm *.aux *.log *.out}
\immediate\write18{ cd "../Soluciones" && rm *.aux *.log *.out}

\expandafter\stop
\fi

\ifcase\condition
\excludecomment{ans}
\or
\includecomment{ans}
\fi

\begin{document}
\maketitle

\begin{prob}
    Sean \(x_n\) e \(y_n\) dos sucesiones tales que \(x_n\leq y_n\) para todo \(n\in\set{N}\). Demuestre que si \(x_n\) no es acotada superiormente entonces \(y_n\) no es acotada superiormente.
\end{prob}

\begin{ans}
    \begin{sol}
        Se recuerda la definición de que \(x_n\) no este acotada superiormente
        \begin{equation*}
            \forall M\in\set{R}\exists n\in\set{N}:x_n>M
        \end{equation*}
        Sea \(M\in\set{R}\), por definición existe un \(n\in\set{N}\) tal que \(M<x_n\leq y_n\), por lo que \(y_n\) no es acotada superiormente.
    \end{sol}
\end{ans}


\begin{prob}
    Demuestre que para todo par de números reales \(x,y\) distintos existe un racional \(z\) tal que \(x<z<y\).
    \textit{Hint: Usar propiedad arquimediana y parte entera.}
\end{prob}

\begin{ans}
    \begin{sol}
        Sea \(w=\frac{x+y}2=x+\frac{y-x}2\), se ve que se tienen las siguientes desigualdades:
        \begin{equation*}
            \floor{nx}\leq nx<\floor{nx}+1\leq nx+1
        \end{equation*}
        Y al dividir por \(n\) se tiene
        \begin{equation*}
            \frac{\floor{nx}}{n}\leq x<\frac{\floor{nx}+1}{n}\leq x+\frac1{n}
        \end{equation*}
        Ahora, por propiedad arquimediana se tiene que existe un \(n\in\set{N}\) tal que \(\frac1n<y-x\), por lo que \(x+\frac1n<y\), entonces \(x<\frac{\floor{nx}+1}n<y\) y \(\frac{\floor{nx}+1}n\in\set{Q}\).
    \end{sol}
\end{ans}


\begin{prob}
    Demuestre que la sucesión
    \begin{equation*}
        x_n=\sum_{k=1}^n\frac1k
    \end{equation*}
    cumple que \(x_{2^n}\geq\frac{n+1}2\) para todo \(n\in\set{N}\).
\end{prob}

\begin{ans}
    \begin{sol}
        Por inducción, para \(n=0\) se tiene que \(x_1=1\geq\frac{0+1}2\). Para el paso inductivo, se ve lo siguiente:
        \begin{align*}
            x_{2^n} & =\sum_{k=1}^{2^n}\frac1k                                     \\
                    & =\sum_{k=1}^{2^{n-1}}\frac1k+\sum_{k=2^{n-1}+1}^{2^n}\frac1k \\
                    & =x_{2^{n-1}}+\sum_{k=2^{n-1}+1}^{2^n}\frac1k                 \\
                    & \geq \frac{(n-1)+1}2+\sum_{k=2^{n-1}+1}^{2^n}\frac1k         \\
                    & \geq \frac{(n-1)+1}2+\sum_{k=2^{n-1}+1}^{2^n}\frac1{2^n}     \\
                    & \geq \frac{(n-1)+1}2+2^{n-1}\cdot\frac1{2^n}                 \\
                    & \geq \frac{(n-1)+1}2+\frac12                                 \\
                    & \geq \frac{n+1}2                                             \\
        \end{align*}
    \end{sol}
\end{ans}


\begin{prob}
    Demuestre que todo sucesión creciente y no acotada \(x_n\) cumple que su límite existe y \(\lim_{n\rightarrow\infty}x_n=\infty\).
\end{prob}

\begin{ans}
    \begin{sol}
        Se nota que toda sucesión creciente es acotada inferiormente, por lo que \(x_n\) es no acotada superiormente. Se recuerda la definición de ser creciente y de no acotada superiormente:
        \begin{align*}
            \text{Creciente:}                & \qquad\forall n\in\set{N} \qquad x_n\leq x_{n+1}          \\
            \text{No acotada superiormente:} & \qquad\forall M\in\set{R}\exists n\in\set{N} \qquad x_n>M
        \end{align*}
        Por lo que dado un \(R>0\) existe un \(n_0\in\set{N}\) tal que \(x_{n_0}>R\), más aún para \(n\geq n_0\) se tiene que \(x_n\geq x_{n_0}>R\). Por lo que se tiene que \(\lim_{n\rightarrow\infty}x_n=\infty\).
    \end{sol}
\end{ans}



\begin{prob}
    Sea \(x_n\) una sucesión se denota \(s_n\) a la sucesión de las sumas parciales:
    \begin{equation*}
        s_n=\sum_{k\leq n}x_k
    \end{equation*}
    Demuestre que si todos los términos de \(x_n\) son positivos, entonces \(s_n\) es creciente. Demuestre también que si para todo \(n\in\set{N}\) se tiene que \(x_n>\varepsilon\) para algún \(\varepsilon>0\), entonces \(\lim_{n\rightarrow\infty}s_n=\infty\).
\end{prob}

\begin{ans}
    \begin{sol}
        Se ve que \(s_{n+1}-s_n=x_{n+1}>0\) por lo que \(s_n\) es creciente.

        Si \(x_n>\varepsilon\) entonces \(s_n=\sum_{k\leq n}x_k\geq\sum_{k\leq n}\varepsilon\geq n\varepsilon\), por ayudantía anterior se tiene que \(\lim_{n\rightarrow\infty}n\varepsilon=\infty\) y por otra ayudantía anterior se tiene que \(\lim_{n\rightarrow\infty}s_n=\infty\).
    \end{sol}
\end{ans}

\end{document}
