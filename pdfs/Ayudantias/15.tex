\documentclass{ayudantia}
\usepackage{multicol}
\title{Ayudantía 15}
\date{2020-10-20}
\course{MAT1106 --- Introducción al Cálculo}

% Comment for final compile
\ifx\condition\undefined
\def\condition{1}
\fi

% TODO Change some problems and add more limit algebra problems specifically some that use adding zeroes to use the triangle inequality

\ifx\condition\undefined
\immediate\write18{ pdflatex -synctex=1 -output-directory="../Enunciados" --jobname="Enunciado\jobname" "\gdef\string\condition{0} \string\input\space\jobname"} 
\immediate\write18{ pdflatex -synctex=1 -output-directory="../Soluciones" --jobname="Solucion\jobname" "\gdef\string\condition{1} \string\input\space\jobname"} 

\immediate\write18{ cd "../Enunciados" && rm *.aux *.log *.out}
\immediate\write18{ cd "../Soluciones" && rm *.aux *.log *.out}

\expandafter\stop
\fi

\ifcase\condition
\excludecomment{ans}
\or
\includecomment{ans}
\fi

\begin{document}
\maketitle


\begin{prob}
    \begin{enumerate}[label=(\alph*)]
        \item Demuestre que \(\lim_{n\rightarrow\infty}\sqrt{1-\frac1{n^2}}=1\)
        \item Demuestre que \(\lim_{n\rightarrow\infty}\cos\paren{\frac1n}=1\)
        \item Demuestre que \(\lim_{n\rightarrow\infty}n\sin\paren{\frac1n}=1\)
    \end{enumerate}
\end{prob}

\begin{ans}
    \begin{sol}

    \end{sol}
\end{ans}



\begin{prob}
    Sea \(x_n\) una sucesión, se define \(c_n\) de la siguiente forma
    \begin{equation*}
        c_n=\frac{x_1+x_2+\ldots+x_n}n.
    \end{equation*}
    Demuestre que si \(x_n\rightarrow L\) entonces \(c_n\rightarrow L\)
\end{prob}

\begin{ans}
    \begin{sol}

    \end{sol}
\end{ans}



\begin{prob}
    Sea \(x_n\) una sucesión convergente y \(\varepsilon>0\), demuestre que existe una subsucesión \(x_{n_k}\) tal que para todo \(k\in\set{N}\) se tiene
    \begin{equation*}
        \abs{x_{n_k}-x_{n_{k+1}}}<\varepsilon.
    \end{equation*}
\end{prob}

\begin{ans}
    \begin{sol}

    \end{sol}
\end{ans}


\begin{prob}
    Sea \(x_n\) una sucesión tal que \(x_n\rightarrow L\) con \(L\neq0\), demuestre que eventualmente \(x_n\) tiene el mismo signo.
\end{prob}

\begin{ans}
    \begin{sol}

    \end{sol}
\end{ans}

\end{document}




