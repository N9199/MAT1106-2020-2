\documentclass{ayudantia}
\usepackage{multicol}
\title{Ayudantía 15}
\date{2020-10-22}
\course{MAT1106 --- Introducción al Cálculo}

% Comment for final compile
%\ifx\condition\undefined
%\def\condition{1}
%\fi

\ifx\condition\undefined
\immediate\write18{ pdflatex -synctex=1 -output-directory="../Enunciados" --jobname="Enunciado\jobname" "\gdef\string\condition{0} \string\input\space\jobname"} 
\immediate\write18{ pdflatex -synctex=1 -output-directory="../Soluciones" --jobname="Solucion\jobname" "\gdef\string\condition{1} \string\input\space\jobname"} 

\immediate\write18{ cd "../Enunciados" && rm *.aux *.log *.out}
\immediate\write18{ cd "../Soluciones" && rm *.aux *.log *.out}

\expandafter\stop
\fi

\ifcase\condition
\excludecomment{ans}
\or
\includecomment{ans}
\fi

\begin{document}
\maketitle


\begin{prob}
    \begin{enumerate}[label=(\alph*)]
        \item Demuestre que \(\lim_{n\rightarrow\infty}\sqrt{1-\frac1{n^2}}=1\).
        \item Demuestre que \(\lim_{n\rightarrow\infty}\cos\paren{\frac1n}=1\).
        \item Demuestre que \(\lim_{n\rightarrow\infty}n\sin\paren{\frac1n}=1\).
        \item Encuentre el límite de \(\paren{\frac{3n-5}{4+3n}}^5\).
    \end{enumerate}
\end{prob}

\begin{ans}
    \begin{sol}

    \end{sol}
\end{ans}


\begin{prob}
    Sean \(x_n\) e \(y_n\) sucesiones tales que \(x_n\rightarrow x\) e \(y_n\rightarrow y\). Demuestre lo siguiente:
    \begin{enumerate}[label=(\alph*)]
        \item \((x_n+y_n)\rightarrow x+y\)
        \item \((x_ny_n)\rightarrow xy\)
    \end{enumerate}
\end{prob}

\begin{ans}
    \begin{sol}

    \end{sol}
\end{ans}


\begin{prob}
    Sean \(p(x)=a_kx^k+\ldots+a_0\) y \(q(x)=b_jx^j+\ldots+b_0\), con \(a_k\) y \(b_j\) distintos de \(0\).
    \begin{enumerate}
        \item Demuestre que si \(k>j\)
        \begin{equation*}
            \lim_{n\rightarrow\infty}\frac{p(n)}{q(n)}=\pm\infty
        \end{equation*}
        \item Demuestre que si \(k=j\)
        \begin{equation*}
            \lim_{n\rightarrow\infty}\frac{p(n)}{q(n)}=\frac{a_k}{b_j}
        \end{equation*}
        \item Demuestre que si \(k<j\)
        \begin{equation*}
            \lim_{n\rightarrow\infty}\frac{p(n)}{q(n)}=0
        \end{equation*}
    \end{enumerate}
\end{prob}

\begin{ans}
    \begin{sol}

    \end{sol}
\end{ans}



\begin{prob}
    Sea \(x_n\) una sucesión convergente y \(\varepsilon>0\), demuestre que existe una subsucesión \(x_{n_k}\) tal que para todo \(k\in\set{N}\) se tiene
    \begin{equation*}
        \abs{x_{n_k}-x_{n_{k+1}}}<\varepsilon.
    \end{equation*}    
\end{prob}    

\begin{ans}
    \begin{sol}

    \end{sol}    
\end{ans}    

\begin{prob}
    Sea \(x_n\) una sucesión. Definimos \(s_n=\sum_{k=1}^nx_k\). Asuma que \(s_n\rightarrow L\) y que \(x_n\) es siempre positiva. Definimos
    \begin{equation*}
        r_n=\lim_{m\rightarrow\infty}\sum_{k=n+1}^mx_k.
    \end{equation*}
    \begin{enumerate}[label=(\alph*)]
        \item Encuentre \(r_n\) de manera explicita.
        \item Demuestre que \(r_n\rightarrow 0\).
    \end{enumerate}
\end{prob}

\begin{ans}
    \begin{sol}

    \end{sol}
\end{ans}

\end{document}




