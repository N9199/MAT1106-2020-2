\documentclass{ayudantia}
\usepackage{multicol}
\title{Ayudantía 09}
\date{2020-09-29}
\course{MAT1106 --- Introducción al Cálculo}

% Comment for final compile
%\ifx\condition\undefined
%\def\condition{1}
%\fi

\ifx\condition\undefined
\immediate\write18{ pdflatex -synctex=1 -output-directory="../Enunciados" --jobname="Enunciado\jobname" "\gdef\string\condition{0} \string\input\space\jobname"} 
\immediate\write18{ pdflatex -synctex=1 -output-directory="../Soluciones" --jobname="Solucion\jobname" "\gdef\string\condition{1} \string\input\space\jobname"} 

\immediate\write18{ cd "../Enunciados" && rm *.aux *.log *.out}
\immediate\write18{ cd "../Soluciones" && rm *.aux *.log *.out}

\expandafter\stop
\fi

\ifcase\condition
\excludecomment{ans}
\or
\includecomment{ans}
\fi

\begin{document}
\maketitle

\begin{prob}
    Determine si las siguientes proposiciones son verdaderas o falsas. Si es verdadero demuestre, en caso contrario de contraejemplo.
    \begin{enumerate}[label=(\alph*)]
        \item Si \(\{x_n\}_{n\in\set{N}}\) y \(\{y_n\}_{n\in\set{N}}\) son acotadas, entonces \(\{(x+y)_n\}_{n\in\set{N}}\) es acotada.
        \item Si \(\{x_n\}_{n\in\set{N}}\) y \(\{y_n\}_{n\in\set{N}}\) son acotadas, entonces \(\{(xy)_n\}_{n\in\set{N}}\) es acotada.
        \item Si \(\{x_n\}_{n\in\set{N}}\) es acotada y \(x_n\neq0\) para todo \(n\), entonces \(\{(\frac1x)_n\}_{n\in\set{N}}\) es acotada.
    \end{enumerate}
\end{prob}

\begin{ans}
    \begin{sol}
        \begin{enumerate}[label=(\alph*)]
            \item Se tiene que \(\abs{x_n}\leq M_1,\abs{y}\leq M_2\) ve que \(\abs{x_n+y_n}\leq\abs{x_n}+\abs{y_n}\leq M_1+M_2\), por lo que es acotada.
            \item Se tiene que \(\abs{x_n}\leq M_1,\abs{y}\leq M_2\) ve que \(\abs{x_n\cdot y_n}=\abs{x_n}\cdot\abs{y_n}\leq M_1\cdot M_2\), por lo que es acotada.
            \item Se ve que \(x_n=\frac{(-1)^n}n\) es un contraejemplo.
        \end{enumerate}
    \end{sol}
\end{ans}


\begin{prob}
    Demuestre que \(x_n=\sqrt{n}\) no está acotada.
\end{prob}

\begin{ans}
    \begin{sol}
        Si es acotada se tiene que \(x^2_n=n\) es acotada por el problema 1 lo que es una contradicción.
    \end{sol}
\end{ans}


\begin{prob}
    Demuestre que las siguientes propiedades son equivalentes:
    \begin{enumerate}[label=(\alph*)]
        \item Para todo \(x\in\set{R}\), existe un \(n\in\set{N}\) tal que \(x<n\).
        \item Para todo \(\varepsilon>0\) existe, un \(n\in\set{N}\) tal que \(\frac1n<\varepsilon\).
        \item Para todos \(\alpha,\varepsilon\in\set{R}\) con \(\varepsilon>0\), existe un \(n\in\set{N}\) tal que \(\alpha<n\varepsilon\).
    \end{enumerate}
\end{prob}

\begin{ans}
    \begin{sol}
        \begin{itemize}
            \item[\(a\implies c\):] Sea \(\alpha\in\set{R}\) y \(\varepsilon>0\) se tiene que \(x=\frac\alpha\varepsilon\in\set{R}\), por lo que existe un \(n\in\set{N}\) tal que \(x<n\), o equivalentemente existe un \(n\in\set{N}\) tal que \(\alpha<n\varepsilon\).
            \item[\(c\implies b\):] Sea \(\varepsilon>0\), se toma \(\alpha=1\), por lo que existe un \(n\in\set{N}\) tal que \(\alpha<n\varepsilon\), o equivalentemente \(\frac1n<\varepsilon\).
            \item[\(b\implies a\)] Se tienen dos casos, si \(x\leq0\) se tiene que \(n=1\) cumple, si \(x>0\) se toma \(\varepsilon=x\) y se tiene que existe \(n\in\set{N}\) tal que \(\frac1n<\varepsilon\), que es equivalente a \(x<n\).
        \end{itemize}
    \end{sol}
\end{ans}


\begin{prob}
    Considere los intervalos de la forma \(J_n=[3,3+\frac1n]\) con \(n\in\set{N}\). Use la pregunta anterior para demostrar que \(\bigcap_{n\in\set{N}}J_n=\{3\}\).
\end{prob}

\begin{ans}
    \begin{sol}
        Sea \(S=\bigcap_{n\in\set{N}}J_n\), si existe \(a\in S\) tal que \(a\neq 3\), se tiene que \(a>3\), por lo que \(\varepsilon=a-3\) nos da que existe \(n\in\set{N}\) tal que \(3+\frac1n<3+\varepsilon=3+a-3=a\), lo que es una contradicción.
    \end{sol}
\end{ans}


\begin{prob}
    Sean \(\alpha,\beta\in\set{R}\) tal que \(\beta>0\). Dado \(x_0=a\) tal que \(\alpha+\beta\cdot a\geq0\), se define recursivamente \(x_{n+1}=\sqrt{\alpha+\beta x_n}\). Demuestre que es una sucesión monótona y acotada.
\end{prob}

\begin{ans}
    \begin{sol}
        Se ve lo siguiente:
        \begin{align*}
            x_n                     & \leq x_{n+1}                     \\
            \beta x_n               & \leq \beta x_{n+1}               \\
            \alpha+\beta x_n        & \leq \alpha+\beta x_{n+1}        \\
            \sqrt{\alpha+\beta x_n} & \leq \sqrt{\alpha+\beta x_{n+1}} \\
            x_{n+1}                 & \leq x_{n+2}                     \\
        \end{align*}
        Se nota, que esto funciona para \(\geq\), ahora por tricotomía se tiene que \(x_1\leq x_2\) o \(x_2\geq x_1\), por lo que usando ese paso inductivo se tiene que \(x_n\) es monótona.
        Para ver que es acotada se tienen dos casos, si es decreciente es trivialmente acotada, \(x_n\geq\min(0,x_0)\) y \(x_n\leq x_0\). Si es creciente, se ve lo siguiente
        \begin{align*}
            x_n & =\sqrt{\alpha+\beta x_{n-1}} \\
            x_n^2 & =\alpha+\beta x_{n-1} \\
            x_n\cdot x_{n-1} & \leq x_n^2=\alpha+\beta x_{n-1} \\
            x_n&\leq\frac\alpha{x_{n-1}}+\beta\leq\frac\alpha{x_1}+\beta
        \end{align*}
        Con lo que tenemos que es acotada.
    \end{sol}
\end{ans}


\end{document}