\documentclass{ayudantia}
\usepackage{multicol}
\title{Ayudantía 11}
\date{2020-10-05}
\course{MAT1106 --- Introducción al Cálculo}

% Comment for final compile
%\ifx\condition\undefined
%\def\condition{1}
%\fi

\ifx\condition\undefined
\immediate\write18{ pdflatex -synctex=1 -output-directory="../Enunciados" --jobname="Enunciado\jobname" "\gdef\string\condition{0} \string\input\space\jobname"} 
\immediate\write18{ pdflatex -synctex=1 -output-directory="../Soluciones" --jobname="Solucion\jobname" "\gdef\string\condition{1} \string\input\space\jobname"} 

\immediate\write18{ cd "../Enunciados" && rm *.aux *.log *.out}
\immediate\write18{ cd "../Soluciones" && rm *.aux *.log *.out}

\expandafter\stop
\fi

\ifcase\condition
\excludecomment{ans}
\or
\includecomment{ans}
\fi

\begin{document}
\maketitle

\begin{prob}
    Sea \(x_n\) una sucesión. Demuestre que \(\lim_{n\rightarrow\infty}x_n=\infty\) si y solo si para todo \(k>0\) \(\lim_{n\rightarrow\infty}k\cdot x_n=\infty\).
\end{prob}

\begin{ans}
    \begin{sol}

\end{sol}
\end{ans}


\begin{prob}
    Demuestre que si \(\lim_{n\rightarrow\infty}x_n=\infty\), entonces \(\frac1{x_n}\) está acotada inferiormente.
\end{prob}

\begin{ans}
    \begin{sol}

\end{sol}
\end{ans}


\begin{prob}
    Sea \(x_n=\frac1{\sqrt{n^3}-\sqrt{n^3-1}}\), demuestre que \(\lim_{n\rightarrow\infty}x_n=\infty\).
\end{prob}

\begin{ans}
    \begin{sol}

\end{sol}
\end{ans}


\begin{prob}
    Sea \(L_n\) definida como
    \begin{equation*}
        L_n=\begin{cases}
            2 &\text{si }n=1\\
            1 &\text{si }n=2\\
            L_{n-1}+L_{n-2} &\text{si }n>2\\
        \end{cases}
    \end{equation*}
    Demuestre que \(L_n\rightarrow\infty\) 
\end{prob}

\begin{ans}
    \begin{sol}

\end{sol}
\end{ans}



\begin{prob}
    Sea \(x_n=\frac1{n^k}\binom{n}{m}\) con \(k,m\in\set{N}\) y \(k< m\), demuestre que \(x_n\rightarrow\infty\).
\end{prob}

\begin{ans}
    \begin{sol}

\end{sol}
\end{ans}

\end{document}

