\documentclass{ayudantia}
\usepackage{multicol}
\title{Ayudantía 12}
\date{2020-10-05}
\course{MAT1106 --- Introducción al Cálculo}

% Comment for final compile
%\ifx\condition\undefined
%\def\condition{1}
%\fi

\ifx\condition\undefined
\immediate\write18{ pdflatex -synctex=1 -output-directory="../Enunciados" --jobname="Enunciado\jobname" "\gdef\string\condition{0} \string\input\space\jobname"} 
\immediate\write18{ pdflatex -synctex=1 -output-directory="../Soluciones" --jobname="Solucion\jobname" "\gdef\string\condition{1} \string\input\space\jobname"} 

\immediate\write18{ cd "../Enunciados" && rm *.aux *.log *.out}
\immediate\write18{ cd "../Soluciones" && rm *.aux *.log *.out}

\expandafter\stop
\fi

\ifcase\condition
\excludecomment{ans}
\or
\includecomment{ans}
\fi

\begin{document}
\maketitle

\begin{prob}
    Sea \(x_n\) una sucesión. Demuestre que \(\lim_{n\rightarrow\infty}x_n=\infty\) si y solo si para todo \(k>0\) \(\lim_{n\rightarrow\infty}k\cdot x_n=\infty\).
\end{prob}

\begin{ans}
    \begin{sol}
        \(\impliedby\) Es trivial tomando \(k=1\).

        \(\implies\) Sea \(R>0\), se nota que \(\frac{R}k>0\), por lo que existe un \(n_0\in\set{N}\) tal que para \(n\geq n_0\) se tiene que \(x_n>\frac{R}k\), o equivalentemente \(k\cdot x_n>R\), lo que nos da que \(\lim_{n\rightarrow\infty}\).
    \end{sol}
\end{ans}


\begin{prob}
    Demuestre que si \(\lim_{n\rightarrow\infty}x_n=\infty\)y \(x_n\neq0\), entonces \(\frac1{x_n}\) está acotada inferiormente.
\end{prob}

\begin{ans}
    \begin{sol}
        Sea \(R=1>0\), se tiene que existe un \(n_0\in\set{N}\) tal que para \(n\geq n_0\) se tiene \(x_n>1>0\), más específicamente se tiene que \(1>\frac1{x_n}>0\). Ahora, sea \(m=\min(\{0\}\cup\{x_n:n<n_0\})\)\footnote{Esto está bien definido ya que \(\{0\}\cup\{x_n:n<n_0\}\) es un conjunto no vacío y es finito.}, se nota que para \(n<n_0\) \(x_n>m\) y que para \(n\geq n_)\) se tiene que \(x_n>0\geq m\), por lo que se tiene que \(x_n\) está acotada inferiormente por \(m\).
    \end{sol}
\end{ans}


\begin{prob}
    Sea \(x_n=\frac1{\sqrt{n^3}-\sqrt{n^3-1}}\), demuestre que \(\lim_{n\rightarrow\infty}x_n=\infty\).
\end{prob}

\begin{ans}
    \begin{sol}
        Se ve la siguiente factorización:
        \begin{align*}
            \frac1{\sqrt{n^3}-\sqrt{n^3-1}}&=\frac1{\sqrt{n^3}-\sqrt{n^3-1}}\cdot\frac{\sqrt{n^3}+\sqrt{n^3-1}}{\sqrt{n^3}+\sqrt{n^3-1}}\\
            &=\sqrt{n^3}+\sqrt{n^3-1}\\
        \end{align*}
        Como \(\sqrt{n^3-1}\geq0\) y \(\sqrt{n^3}\geq n\), se tiene que \(x_n\geq n\), y por ayudantía anterior al tenerse que \(\lim_{n\rightarrow}n=\infty\) se tiene que \(\lim_{n\rightarrow\infty}x_n=\infty\).
    \end{sol}
\end{ans}


\begin{prob}
    Sea \(L_n\) definida como
    \begin{equation*}
        L_n=\begin{cases}
            2               & \text{si }n=1 \\
            1               & \text{si }n=2 \\
            L_{n-1}+L_{n-2} & \text{si }n>2 \\
        \end{cases}
    \end{equation*}
    Demuestre que \(L_n\rightarrow\infty\)
\end{prob}

\begin{ans}
    \begin{sol}
        Se demuestra por inducción que \(L_n\geq n-1\), para \(n<4\) se ve lo siguiente:
        \begin{align*}
            L_1&=2\geq0\\
            L_2&=1\geq1\\
            L_3&=3\geq2\\
        \end{align*}
        Luego para \(L_n\) se ve
        \begin{align*}
            L_n&=L_{n-1}+L_{n-2}\\
            L_n&\geq n-2+n-3\\
            L_n&\geq 2n-5\\
            L_n&\geq n-1,
        \end{align*}
        la última desigualdad se tiene porque \(n\geq 4\). Ahora por ayudantia anterior se tiene que \(\lim_{n\rightarrow\infty}L_n=\infty\).
    \end{sol}
\end{ans}

\end{document}

