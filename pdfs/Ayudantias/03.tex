\documentclass{ayudantia}
\usepackage{multicol}
\title{Ayudantía 03}
\date{2020-08-25}
\course{MAT1106 --- Introducción al Cálculo}

% Comment for final compile
% \ifx\condition\undefined
% \def\condition{1}
% \fi

\ifx\condition\undefined
\immediate\write18{ pdflatex -synctex=1 -output-directory="../Enunciados" --jobname="Enunciado\jobname" "\gdef\string\condition{0} \string\input\space\jobname"} 
\immediate\write18{ pdflatex -synctex=1 -output-directory="../Soluciones" --jobname="Solucion\jobname" "\gdef\string\condition{1} \string\input\space\jobname"} 

\immediate\write18{ cd "../Enunciados" && rm *.aux *.log *.out}
\immediate\write18{ cd "../Soluciones" && rm *.aux *.log *.out}

\expandafter\stop
\fi

\ifcase\condition
\excludecomment{ans}
\or
\includecomment{ans}
\fi

\begin{document}
\maketitle

\begin{prob}
    ¿Bajo qué condiciones \(\abs{x+y}=\abs{x}+\abs{y}\)?
\end{prob}

\begin{ans}
    \begin{sol}
        Se nota que si \(x\) e \(y\) son positivos, entonces \(\abs{x+y}=\abs{x}+\abs{y}\), ahora si \(x\) e \(y\) ambos son negativos, se tiene que \(-x\) y \(-y\) son positivos, luego
        \begin{align*}
            \abs{x+y}&=\abs{-(x+y)}\\
            &=\abs{(-x)+(-y)}\\
            &=\abs{(-x)}+\abs{(-y)}\\
            &=\abs{x}+\abs{y}
        \end{align*}
        Ahora, si el signo de \(x\) es distinto al de \(y\), se asume s.p.d.g. que \(x>0\) y ambos no cero, entonces
        \begin{align*}
            \abs{x}+\abs{y}&=\abs{x}+\abs{-y}\\
            &=\abs{x+(-y)}
        \end{align*}
        Ahora, se tiene que \(\abs{x+y}=\abs{x}+\abs{y}\) si y solo si \(\abs{x-y}=\abs{x+y}\), pero esto implica que \((x-y)^2=(x+y)^2\) por lo que \(4xy=0\), como \(x>0\) y \(4>0\) se tiene que \(y=0\), pero se asumió que \(y<0\), por lo que se tiene que \(\abs{x+y}\neq\abs{x}+\abs{y}\). Por último si \(x\) o \(y\), se puede asumir s.p.d.g. que \(y=0\)\footnote{se permuta \(x\) con \(y\).} y por lo tanto se tiene que
        \begin{align*}
            \abs{x+y}&=\abs{x+0}\\
            &=\abs{x}\\
            &=\abs{x}+0\\
            &=\abs{x}+\abs{y}\\
        \end{align*}
        Con eso se ven todos los casos.
    \end{sol}
\end{ans}


\begin{prob}
    Muestre que \(\abs{x_1+\dots+x_n}\leq\abs{x_1}+\dots+\abs{x_n}\) para cualesquiera \(x_1,\dots,x_n\).
\end{prob}

\begin{ans}
    \begin{sol}
        Por inducción sobre \(n\), si \(n=1\), se tiene trivialmente que \(\abs{x_1}\leq\abs{x_1}\). Ahora, se asume para \(k=n\). Sean \(x_1,\dots,x_{n+1}\) reales cualesquiera, entonces se tiene lo siguiente:
        \begin{align*}
            \abs{(x_1+\dots+x_n)+x_{n+1}}&\leq\abs{x_1+\dots+x_n}+\abs{x_{n+1}}\\
            &\leq\abs{x_1+\dots+x_n}+\abs{x_{n+1}}\\
            &\leq\abs{x_1}+\dots+\abs{x_n}+\abs{x_{n+1}}
        \end{align*}
        Demostrando lo pedido.
    \end{sol}
\end{ans}


\begin{prob}
    Encuentre el conjunto solución de las siguientes expresiones:
    \begin{multicols}{2}
        \begin{enumerate}
            \item \(\ds\frac{x^2-1}{x^2-4}\leq 0\)
            \item \(\ds\abs{\frac{5x-2}{3x+1}}> 7\)
        \end{enumerate}
    \end{multicols}
\end{prob}

\begin{ans}
    \begin{sol}
        \begin{enumerate}
            \item Se factoriza la expresión a la derecha y se ve la siguiente desigualdad
            \begin{equation*}
                \frac{(x-1)(x+1)}{(x-2)(x+2)}\leq0
            \end{equation*}
            Se nota que si \(x=-2\) o \(x=2\), la expresión no está bien definida. Ahora, a través del uso de la siguiente tabla se observarán los posibles casos:
            \begin{center}
                \begin{tabular}{c|c|c|c|c|c|}
                    &\((-\infty,-2)\)&\((-2,-1)\)&\((-1,1)\)&\((1,2)\)&\((2,\infty)\)\\\hline
                    \(x+2\)&-&+&+&+&+\\
                    \(x+1\)&-&-&+&+&+\\
                    \(x-1\)&-&-&-&+&+\\
                    \(x-2\)&-&-&-&-&+\\
                    \(\ds\frac{(x-1)(x+1)}{(x-2)(x+2)}\)&+&-&+&-&+\\
                \end{tabular}
            \end{center}
            Con lo anterior se ve que la expresión es estrictamente menor a \(0\) en \((-2,-1)\cup(1,2)\), para el caso donde la expresión es igual a \(0\) se nota que la expresión es cero si \(x=1\) o \(x=-1\), por lo que el conjunto solución es \((-2,-1]\cup[1,2)\).
            \item Se nota que si \(x>-\frac13\) se tiene que \(3x+1>0\), similarmente si \(x<-\frac13\) se tiene que \(3x+1<0\), juntando eso con que \(\abs{\frac{5x-2}{3x+1}}>7\) si y solo si \(\frac{5x-2}{3x+1}>7\) o \(\frac{5x-2}{3x+1}<-7\), se tienen cuatro casos:
            \begin{enumerate}[label={Caso \arabic*:}]
                \item \(x>-\frac13\) y \(\frac{5x-2}{3x+1}>7\), dado eso se tiene la siguiente cadena de \(\iff\)
                \begin{align*}
                    \frac{5x-2}{3x+1}>7&\iff 5x-2>7*\paren{3x+1}\\
                    &\iff 5x-2>21x+7\\
                    &\iff -9>16x\\
                    &\iff -\frac9{16}>x\\
                \end{align*}
                Por lo que \(x\in(-\frac13,-\frac9{16})\).
                \item \(x<-\frac13\) y \(\frac{5x-2}{3x+1}>7\), dado eso se tiene la siguiente cadena de \(\iff\)
                \begin{align*}
                    \frac{5x-2}{3x+1}>7&\iff 5x-2<7*\paren{3x+1}\\
                    &\iff 5x-2<21x+7\\
                    &\iff -9<16x\\
                    &\iff -\frac9{16}<x\\
                \end{align*}
                Por lo que \(x\in(-\frac9{16},-\frac13)\).
                \item \(x>-\frac13\) y \(\frac{5x-2}{3x+1}<-7\), dado eso se tiene la siguiente cadena de \(\iff\)
                \begin{align*}
                    \frac{5x-2}{3x+1}<-7&\iff 5x-2<-7*\paren{3x+1}\\
                    &\iff 5x-2<-21x-7\\
                    &\iff 26x<-5\\
                    &\iff x<-\frac5{26}\\
                \end{align*}
                Por lo que \(x\in(-\frac13,-\frac5{26})\).
                \item \(x<-\frac13\) y \(\frac{5x-2}{3x+1}<-7\), dado eso se tiene la siguiente cadena de \(\iff\)
                \begin{align*}
                    \frac{5x-2}{3x+1}<-7&\iff 5x-2<-7*\paren{3x+1}\\
                    &\iff 5x-2>-21x-7\\
                    &\iff 26x>-5\\
                    &\iff x>-\frac5{26}\\
                \end{align*}
                Por lo que \(x\in(-\frac5{26},-\frac13)\).
            \end{enumerate}
            Juntando todos lo casos, y notando que \(-\frac9{16}<-\frac13\) y que \(-\frac13<-\frac5{26}\), se tiene que el conjunto solución es \((-\frac9{16},-\frac13)\cup (-\frac13,-\frac5{26})\)
        \end{enumerate}
    \end{sol}
\end{ans}


\begin{prob}
    Sea \(x>0\). Demuestre que
    \begin{equation*}
        \sqrt{1+x}\leq1+x\leq(1+\sqrt{x})^2
    \end{equation*}
\end{prob}

\begin{ans}
    \begin{sol}
        Se nota que \(1+x>0\), por lo que \(\sqrt{1+x}>0\). Dado eso, se ven las siguientes equivalencias:
        \begin{align*}
            0<\sqrt{1+x}\leq1+x&\iff 0<1+x\leq(1+x)^2
            &\iff 0<1+x\leq1+2x+x^2
            &\iff 1+x\leq1+2x+x^2
            &\iff 0\leq x+x^2
            &\iff 0\leq x(x+1)
        \end{align*}
        Como \(x>0\) se tiene que \(0<x(x+1)\). Y por la cadena de equivalencias se tiene la primera desigualdad. Para la segunda desigualdad, se ven la siguientes equivalencias
        \begin{align*}
            0<1+x\leq(1+\sqrt{x})^2&\iff 1+x\leq1+2\sqrt{x}+x
            &\iff 0\leq 2\sqrt{x}
        \end{align*}
        De nuevo, como \(x>0\) se tiene que \(\sqrt{x}>0\), y por la cadena de equivalencias se tiene lo pedido.
    \end{sol}
\end{ans}


\end{document}