\documentclass{ayudantia}
\usepackage{multicol}
\title{Ayudantía 24}
\date{2020-12-01}
\course{MAT1106 --- Introducción al Cálculo}

% Comment for final compile
%\ifx\condition\undefined
%\def\condition{1}
%\fi

\ifx\condition\undefined
\immediate\write18{ pdflatex -synctex=1 -output-directory="../Enunciados" --jobname="Enunciado\jobname" "\gdef\string\condition{0} \string\input\space\jobname"} 
\immediate\write18{ pdflatex -synctex=1 -output-directory="../Soluciones" --jobname="Solucion\jobname" "\gdef\string\condition{1} \string\input\space\jobname"} 

\immediate\write18{ cd "../Enunciados" && rm *.aux *.log *.out}
\immediate\write18{ cd "../Soluciones" && rm *.aux *.log *.out}

\expandafter\stop
\fi

\ifcase\condition
\excludecomment{ans}
\or
\includecomment{ans}
\fi

\begin{document}
\maketitle

\begin{prob}
    Demuestre que existe una única función \(f:\set{R}\rightarrow\set{R}\) tal que para toda función \(g:\set{R}\rightarrow\set{R}\) se tiene que \(f\circ g=g\circ f\).
\end{prob}

\begin{ans}
    \begin{sol}

    \end{sol}
\end{ans}



\begin{prob}
    Demuestre las siguientes propiedades de funciones:
    \begin{enumerate}
        \item \(f:A\subseteq\set{R}\rightarrow B\set{R}\) es una función biyectiva si y solo si tiene inversa \(f^{-1}:B\subseteq\set{R}\rightarrow A\subseteq\set{R}\) tal que \(\forall x\in B,f\circ f^{-1}(x)=x\) y \(\forall x\in A,f^{-1}\circ f(x)=x\).
        \item Sean \(f:A\subseteq\set{R}\rightarrow B\subseteq\set{R}\) y \(g:B\subseteq\set{R}\rightarrow C\subseteq\set{R}\) funciones biyectivas, entonces \(g\circ f:A\subseteq\set{R}\rightarrow C\subseteq\set{R}\) es biyectiva.
        \item Sea \(f:\set{R}\rightarrow\set{R}\) tal que para toda sucesión \(x_n\rightarrow x\) se tiene que \(\lim_{n\rightarrow\infty}f(x_n)=f(\lim_{n\rightarrow\infty}x_n)=f(x)\), demuestre que si \(A\subseteq\set{R}\) es acotado entonces \(f(A)\) es acotado.
    \end{enumerate}
\end{prob}

\begin{ans}
    \begin{sol}

    \end{sol}
\end{ans}



\begin{prob}
    Demuestre que \(f:A\rightarrow\set{R}\) donde \(f(x)=\frac{x}{(x-1)(x+1)}\) es biyectiva si \(A\) es
    \begin{itemize}
        \item \((-1,1)\)
        \item \(\set{R}\setminus[-1,1]\)
    \end{itemize}
\end{prob}

\begin{ans}
    \begin{sol}

    \end{sol}
\end{ans}

\end{document}
