\documentclass{ayudantia}
\usepackage{multicol}
\title{Ayudantía 07}
\date{2020-09-08}
\course{MAT1106 --- Introducción al Cálculo}

% Comment for final compile
%\ifx\condition\undefined
%\def\condition{1}
%\fi

\ifx\condition\undefined
\immediate\write18{ pdflatex -synctex=1 -output-directory="../Enunciados" --jobname="Enunciado\jobname" "\gdef\string\condition{0} \string\input\space\jobname"} 
\immediate\write18{ pdflatex -synctex=1 -output-directory="../Soluciones" --jobname="Solucion\jobname" "\gdef\string\condition{1} \string\input\space\jobname"} 

\immediate\write18{ cd "../Enunciados" && rm *.aux *.log *.out}
\immediate\write18{ cd "../Soluciones" && rm *.aux *.log *.out}

\expandafter\stop
\fi

\ifcase\condition
\excludecomment{ans}
\or
\includecomment{ans}
\fi

\begin{document}
\maketitle

\begin{prob}
    Demuestre por inducción que \(n^2\geq n\) para todo \(n\in\set{N}\).
\end{prob}

\begin{ans}
    \begin{sol}
        Usando inducción, para \(n=1\) se tiene que \(1^2\geq 1\). Ahora, para el caso inductivo:
        \begin{align*}
            (n+1)^2&=n^2+2n+1\\
            &\geq n^2+2n+1\\
            &\geq n+2n+1\\
            &\geq n+1
        \end{align*}
        Quedando demostrado el caso inductivo.
    \end{sol}
\end{ans}


\begin{prob}
    Usando la notación \(\{(x+y)_n\}_{n\in\set{N}}\) para la sucesión definida como \((x+y)_n=x_n+y_n\), y \(\{(xy)_n\}_{n\in\set{N}}\) para la sucesión definida como \((xy)_n=x_ny_n\). Determine si las siguientes proposiciones son verdaderas o falsas. Si es verdadero demuestre, en caso contrario de contraejemplo.
    \begin{enumerate}
        \item Si \(\{x_n\}_{n\in\set{N}}\) y \(\{y_n\}_{n\in\set{N}}\) son crecientes, entonces \(\{(x+y)_n\}_{n\in\set{N}}\) es creciente.
        \item Si \(\{x_n\}_{n\in\set{N}}\) y \(\{y_n\}_{n\in\set{N}}\) son crecientes, entonces \(\{(xy)_n\}_{n\in\set{N}}\) es creciente.
        \item Si \(\{x_n\}_{n\in\set{N}}\) y \(\{y_n\}_{n\in\set{N}}\) son monótonas, entonces \(\{(x+y)_n\}_{n\in\set{N}}\) es monótona.
        \item Si \(\{x_n\}_{n\in\set{N}}\) y \(\{y_n\}_{n\in\set{N}}\) son monótonas, entonces \(\{(xy)_n\}_{n\in\set{N}}\) es monótona.
        \item Si \(\{x_n\}_{n\in\set{N}}\) es monótona, entonces \(\{(x^2)_n\}_{n\in\set{N}}\) es creciente.
    \end{enumerate}
\end{prob}

\begin{ans}
    \begin{sol}
        \begin{enumerate}
            \item Es verdadero, se tiene que \(x_n+y_n\leq x_{n+1}+y_n\leq x_{n+1}+y_{n+1}\), por lo que \((x+y)_n\leq (x+y)_{n+1}\).
            \item Es falso, se consideran \(x_n=n\) e \(y_n=-1\).
            \item Es falso, se consideran \(x_n=n^2\) e \(y_n=-n!\), entonces \((x+y)_1=0,(x+y)_2=2,(x+y)_3=3,(x+y)_4=-8\).
            \item Es falso, se consideran \(x_n=n^2\) e \(y_n=(n!)^{-1}\), entonces \((xy)_1=1,(xy)_2=2,(xy)_3=\frac32,(xy)_4=\frac23\).
            \item Es falso, se considera \(x_n=n-2\), entonces \((x^2)_1=1,(x^2)_2=0,(x^2)_3=1\).
        \end{enumerate}
    \end{sol}
\end{ans}


\begin{prob}
    Sean \(n\in\set{N}\), \(\alpha>1\). Demuestre que existe una constante \(C>0\) tal que \(\alpha^n>Cn\) para todo \(n\).
\end{prob}

\begin{ans}
    \begin{sol}
        Se ve lo siguiente:
        \begin{align*}
            \alpha^n&=(1+(\alpha-1))^n\\
            &\geq1+n(\alpha-1)\qquad\text{Por Bernoulli}\\
            &>n(\alpha-1)
        \end{align*}
        Entonces tomando \(C=\alpha-1\) se tiene lo pedido.
    \end{sol}
\end{ans}


\begin{prob}
    Demuestre que la siguiente sucesión es creciente
    \begin{equation*}
        a_n=\paren{1+\frac1n}^n
    \end{equation*}
\end{prob}

\begin{ans}
    \begin{sol}
        Para demostrar que \(a_n\) es creciente se demostrará que \(a_{n+1}/a_n\geq1\). Se ve lo siguiente:
        \begin{align*}
            \frac{a_{n+1}}{a_n}&=\frac{\paren{1+\frac1{n+1}}^{n+1}}{\paren{1+\frac1n}^n}\\
            &=\paren{1+\frac1{n+1}}\paren{\frac{1+\frac1{n+1}}{1+\frac1n}}^n\\
            &=\paren{\frac{n+2}{n+1}}\paren{\frac{\frac{n+2}{n+1}}{\frac{n+1}n}}^n\\
            &=\paren{\frac{n+2}{n+1}}\paren{\frac{n(n+2)}{(n+1)^2}}^n\\
            &=\paren{\frac{n+2}{n+1}}\paren{1-\frac1{(n+1)^2}}^n\\
            &\geq\paren{\frac{n+2}{n+1}}\paren{1-\frac{n}{(n+1)^2}}\\
            &\geq^*\paren{\frac{n+2}{n+1}}\paren{\frac{n+1}{n+2}}\\
            &\geq1\\
        \end{align*}
        Para el \(\geq^*\) se ve que
        \begin{equation*}
            1-\frac{n}{(n+1)^2}=\frac{n^2+n+1}{(n+1)^2}=\frac{(n^2+n+1)(n+2)}{(n+1)^2(n+2)}\geq\frac{(n+1)^3}{(n+1)^2(n+2)}=\frac{n+1}{n+2}
        \end{equation*}
    \end{sol}
\end{ans}


\end{document}