\documentclass{ayudantia}
\usepackage{multicol}
\title{Ayudantía 07}
\date{2020-09-08}
\course{MAT1106 --- Introducción al Cálculo}

% Comment for final compile
%\ifx\condition\undefined
%\def\condition{1}
%\fi

\ifx\condition\undefined
\immediate\write18{ pdflatex -synctex=1 -output-directory="../Enunciados" --jobname="Enunciado\jobname" "\gdef\string\condition{0} \string\input\space\jobname"} 
\immediate\write18{ pdflatex -synctex=1 -output-directory="../Soluciones" --jobname="Solucion\jobname" "\gdef\string\condition{1} \string\input\space\jobname"} 

\immediate\write18{ cd "../Enunciados" && rm *.aux *.log *.out}
\immediate\write18{ cd "../Soluciones" && rm *.aux *.log *.out}

\expandafter\stop
\fi

\ifcase\condition
\excludecomment{ans}
\or
\includecomment{ans}
\fi

\begin{document}
\maketitle

\begin{prob}
    Demuestre por inducción que \(n^2\geq n\) para todo \(n\in\set{N}\).
\end{prob}

\begin{ans}
    \begin{sol}
        \((n+1)^2=n^2+2n+1\geq n+2n+1=3n+1\geq n\)
    \end{sol}
\end{ans}


\begin{prob}
    Usando la notación \(\{(x+y)_n\}_{n\in\set{N}}\) para la sucesión definida como \((x+y)_n=x_n+y_n\), y \(\{(xy)_n\}_{n\in\set{N}}\) para la sucesión definida como \((xy)_n=x_ny_n\). Determine si las siguientes proposiciones son verdaderas o falsas. Si es verdadero demuestre, en caso contrario de contraejemplo.
    \begin{enumerate}
        \item Si \(\{x_n\}_{n\in\set{N}}\) y \(\{y_n\}_{n\in\set{N}}\) son crecientes, entonces \(\{(x+y)_n\}_{n\in\set{N}}\) es creciente.
        \item Si \(\{x_n\}_{n\in\set{N}}\) y \(\{y_n\}_{n\in\set{N}}\) son crecientes, entonces \(\{(xy)_n\}_{n\in\set{N}}\) es creciente.
        \item Si \(\{x_n\}_{n\in\set{N}}\) y \(\{y_n\}_{n\in\set{N}}\) son monótonas, entonces \(\{(x+y)_n\}_{n\in\set{N}}\) es monótona.
        \item Si \(\{x_n\}_{n\in\set{N}}\) y \(\{y_n\}_{n\in\set{N}}\) son monótonas, entonces \(\{(xy)_n\}_{n\in\set{N}}\) es monótona.
        \item Si \(\{x_n\}_{n\in\set{N}}\) es monótona, entonces \(\{(x^2)_n\}_{n\in\set{N}}\) es creciente.
    \end{enumerate}
\end{prob}

\begin{ans}
    \begin{sol}
        \begin{enumerate}
            \item V
            \item F \(x_n=n\), \(y_n=-1\)
            \item F \(x_n=n^2\), \(y_n=-n!\)
            \item F \(x_n=n^2\), \(y_n=(n!)^{-1}\)
            \item F \(x_n=n-2\)
        \end{enumerate}
    \end{sol}
\end{ans}


\begin{prob}
    Sean \(n\in\set{N}\), \(\alpha>1\). Demuestre que existe una constante \(C>0\) tal que \(\alpha^n>Cn\) para todo \(n\).
\end{prob}

\begin{ans}
    \begin{sol}
        Usar Bernoulli
    \end{sol}
\end{ans}


\begin{prob}
    Demuestre que la siguiente sucesión es creciente
    \begin{equation*}
        a_n=\paren{1+\frac1n}^n
    \end{equation*}
\end{prob}

\begin{ans}
    \begin{sol}
        Tomar \(a_n/a_{n+1}\) y usar Bernoulli.
    \end{sol}
\end{ans}


\end{document}