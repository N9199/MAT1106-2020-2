\documentclass{ayudantia}
\usepackage{multicol}
\title{Ayudantía 08}
\date{2020-09-15}
\course{MAT1106 --- Introducción al Cálculo}

% Comment for final compile
%\ifx\condition\undefined
%\def\condition{1}
%\fi

\ifx\condition\undefined
\immediate\write18{ pdflatex -synctex=1 -output-directory="../Enunciados" --jobname="Enunciado\jobname" "\gdef\string\condition{0} \string\input\space\jobname"} 
\immediate\write18{ pdflatex -synctex=1 -output-directory="../Soluciones" --jobname="Solucion\jobname" "\gdef\string\condition{1} \string\input\space\jobname"} 

\immediate\write18{ cd "../Enunciados" && rm *.aux *.log *.out}
\immediate\write18{ cd "../Soluciones" && rm *.aux *.log *.out}

\expandafter\stop
\fi

\ifcase\condition
\excludecomment{ans}
\or
\includecomment{ans}
\fi

\begin{document}
\maketitle

\begin{prob}\textit{(I6 2018)}
    Considere \(x_n=\frac{n!}{n^n}\)
    \begin{enumerate}[label=(\alph*)]
        \item Demuestre que
        \begin{equation*}
            \frac{x_{n+1}}{x_n}\leq\frac12
        \end{equation*}
        \item Demuestre que
        \begin{equation*}
            0\leq x_n\leq\frac1{2^{n-1}}
        \end{equation*}
        para todo \(n\in\set{N}\).
    \end{enumerate}
\end{prob}

\begin{ans}
    \begin{sol}
        \begin{enumerate}[label=(\alph*)]
            \item Desglosar
            \item Usar inducción más la parte anterior
        \end{enumerate}
    \end{sol}
\end{ans}


\begin{prob}
    Demuestre que \(x_n\) es monótona si y solo si todas las subsucesiones también son monótonas.
\end{prob}

\begin{ans}
    \begin{sol}
        \(\implies\) trivial
        
        \(\impliedby\) contrapositiva
    \end{sol}
\end{ans}


\begin{prob}
    Para \(a>0\), se definen las funciones
    \begin{equation*}
        f(x)=x^3-2\qquad\text{y}\qquad g_a(x)=a^3-2+3a^2(x-a)
    \end{equation*}
    \begin{enumerate}[label=(\alph*)]
        \item Demuestre que
        \begin{equation*}
            f(x)-g_a(x)=(x-a)^2(x+2a)
        \end{equation*}
        y concluya que \(f(x)\geq g_a(x)\) para todo \(x\geq 0\).
        \item Ahora, sea \(x_n\) una sucesión tal que \(x_1=2\) y \(x_{n+1}\) cumple
        \begin{equation*}
            g_{x_n}(x_{n+1})=0
        \end{equation*}
        Demuestre que está sucesión es monótona.
    \end{enumerate}
\end{prob}

\begin{ans}
    \begin{sol}
        \begin{enumerate}[label=(\alph*)]
            \item Hacer lo esperado.
            \item Ídem.
        \end{enumerate}
    \end{sol}
\end{ans}


\begin{prob}
    Encuentre condiciones necesarias y suficientes para que una sucesión tenga una cantidad finita de subsucesiones.
\end{prob}

\begin{ans}
    \begin{sol}
        Eventualmente constante
    \end{sol}
\end{ans}


\end{document}