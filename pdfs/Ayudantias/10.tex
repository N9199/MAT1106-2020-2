\documentclass{ayudantia}
\usepackage{multicol}
\title{Ayudantía 10}
\date{2020-10-01}
\course{MAT1106 --- Introducción al Cálculo}

% Comment for final compile
%\ifx\condition\undefined
%\def\condition{1}
%\fi

\ifx\condition\undefined
\immediate\write18{ pdflatex -synctex=1 -output-directory="../Enunciados" --jobname="Enunciado\jobname" "\gdef\string\condition{0} \string\input\space\jobname"} 
\immediate\write18{ pdflatex -synctex=1 -output-directory="../Soluciones" --jobname="Solucion\jobname" "\gdef\string\condition{1} \string\input\space\jobname"} 

\immediate\write18{ cd "../Enunciados" && rm *.aux *.log *.out}
\immediate\write18{ cd "../Soluciones" && rm *.aux *.log *.out}

\expandafter\stop
\fi

\ifcase\condition
\excludecomment{ans}
\or
\includecomment{ans}
\fi

\begin{document}
\maketitle

\begin{prob}
    Sea \(A\subset\set{R}\) un conjunto finito no vacío, demuestre que existen \(m,M\in A\) tales que para todo \(a\in A\) se tiene \(m\leq a\leq M\). \(m\) y \(M\) se denotarán como el mínimo de \(A\)\footnote{\(m=\min A\)} y el máximo de \(A\)\footnote{\(M=\max A\)}, respectivamente.
\end{prob}

\begin{ans}
    \begin{sol}
        Inducción sobre el tamaño del conjunto. Se escribe \(A=\{a\}\cup(A\setminus\{a\})\).
    \end{sol}
\end{ans}


\begin{prob}
    \begin{enumerate}
        \item Sea \(x_n\) una sucesión acotada, demuestre que toda subsucesión es acotada.
        \item Sea \(x_n\) una sucesión monótona no acotada, demuestre que toda subsucesión es no acotada.
        \item Encuentre una sucesión no acotada \(x_n\), tal que tiene al menos una subsucesión acotada. ¿Existe alguna que tenga infinitas subsucesiones acotadas?
    \end{enumerate}
\end{prob}

\begin{ans}
    \begin{sol}
        \begin{enumerate}
            \item Contradicción y definición.
            \item Contradicción, buen orden sobre los indices de subsucesión acotada.
            \item \(x_{2n}=n,x_{2n+1}=0\)
        \end{enumerate}
    \end{sol}
\end{ans}


\begin{prob}
    Sea \(x_n\) una sucesión de números enteros, demuestre que \(x_n\) siempre cumple al menos una de las siguientes propiedades:
    \begin{enumerate}[label=(\alph*)]
        \item Tiene una cantidad finita de términos distintos, en otras palabras el conjunto \(S=\{x_n:n\in\set{N}\}\) es finito.
        \item Es no acotada.
    \end{enumerate}
\end{prob}

\begin{ans}
    \begin{sol}

    \end{sol}
\end{ans}


\begin{prob}
    Sea
    \begin{equation*}
        x_n=\sum_{k=1}^n\frac1k
    \end{equation*}
    Demuestre que para todo \(n\), se tiene que
    \begin{equation*}
        x_{2^n}\geq\frac{n+1}2.
    \end{equation*}
\end{prob}

\begin{ans}
    \begin{sol}

    \end{sol}
\end{ans}


\begin{prob}
    Sea
    \begin{equation*}
        x_n=\sum_{k=1}^n\frac1{k^2}
    \end{equation*}
    ¿Existe algún valor \(n\) tal que \(x_n>2\)?

    \noindent\textit{Hint: Vea que para \(k\geq2\) se tiene \(\frac1{k^2}\leq\frac1{k(k-1)}=\frac1k-\frac1{k-1}\).}
\end{prob}

\begin{ans}
    \begin{sol}

    \end{sol}
\end{ans}


\end{document}