\documentclass{ayudantia}
\usepackage{multicol}
\title{Ayudantía 10}
\date{2020-10-01}
\course{MAT1106 --- Introducción al Cálculo}

% Comment for final compile
%\ifx\condition\undefined
%\def\condition{1}
%\fi

\ifx\condition\undefined
\immediate\write18{ pdflatex -synctex=1 -output-directory="../Enunciados" --jobname="Enunciado\jobname" "\gdef\string\condition{0} \string\input\space\jobname"} 
\immediate\write18{ pdflatex -synctex=1 -output-directory="../Soluciones" --jobname="Solucion\jobname" "\gdef\string\condition{1} \string\input\space\jobname"} 

\immediate\write18{ cd "../Enunciados" && rm *.aux *.log *.out}
\immediate\write18{ cd "../Soluciones" && rm *.aux *.log *.out}

\expandafter\stop
\fi

\ifcase\condition
\excludecomment{ans}
\or
\includecomment{ans}
\fi

\begin{document}
\maketitle

\begin{prob}
    Sea \(A\subset\set{R}\) un conjunto finito no vacío, demuestre que existen \(m,M\in A\) tales que para todo \(a\in A\) se tiene \(m\leq a\leq M\). \(m\) y \(M\) se denotarán como el mínimo de \(A\)\footnote{\(m=\min A\)} y el máximo de \(A\)\footnote{\(M=\max A\)}, respectivamente.
\end{prob}

\begin{ans}
    \begin{sol}
        Se hace Inducción sobre el tamaño de \(A\), si \(\abs{A}=1\), se tiene que \(A=\{a\}\), por lo que \(m=M=a\) cumple lo pedido. Para el paso inductivo, se tiene que \(\abs{A}=n\), como es finito podemos escoger un \(a'\in A\) y reescribimos \(A=\{a'\}\cup(A\setminus\{a'\})\), ahora \(\abs{A\setminus\{a\}}=n-1\), por lo que por hipótesis inductiva tenemos que \(\exists m',M'\in A\setminus\{a\}\quad\forall a\in A\setminus\{a'\}\quad m'\leq a\leq M'\), ahora se nota que \(m=\min(a',m')\leq a\leq M=\max(a',M')\) para todo \(a\in A\), por lo que se tiene lo pedido.
    \end{sol}
\end{ans}


\begin{prob}
    \begin{enumerate}
        \item Sea \(x_n\) una sucesión acotada, demuestre que toda subsucesión es acotada.
        \item Sea \(x_n\) una sucesión monótona no acotada, demuestre que toda subsucesión es no acotada.
        \item Encuentre una sucesión no acotada \(x_n\), tal que tiene al menos una subsucesión acotada. ¿Existe alguna que tenga infinitas subsucesiones acotadas?
    \end{enumerate}
\end{prob}

\begin{ans}
    \begin{sol}
        \begin{enumerate}
            \item Sea \(x_{n_k}\) subsucesión de \(x_n\), por definición se tiene que existe un \(M\in\set{R}\) tal que \(\forall n\in\set{N} \abs{x_n}<M\), como \(\{n_k:k\in\set{N}\}\subset\set{N}\) se tiene que \(\forall k\in \set{N}\abs{x_{n_k}}\leq M\).
            \item Sea \(x_{n_k}\) subsucesión de \(x_n\), se tiene que \(x_{n_k}\) ``hereda'' la monotonía de \(x_n\), ahora s.p.d.g. \(x_n\) es creciente por lo que es acotada inferiormente, por lo que no es acotada superiormente\footnote{Si lo fuera \(x_n\) sería una sucesión acotada}. Dicho eso, sea \(M\in\set{R}\), por lo anterior se tiene que existe un \(n_0\in\set{N}\) tal que \(x_{n_0}>M\), sea \(S_{n_0}={k\in\set{N}:n_k\geq n_0}\), se ve que \(S_{n_0}\) es un subconjunto no vacío de \(\set{N}\), por lo que tiene un mínimo, que se denotara \(k_0\), luego se tiene que \(n_{k_0}\geq n_0\) por lo que \(x_{n_{k_0}}\geq x_{n_0}>M\), por lo que \(x_{n_k}\) no es acotada.
            \item Se ve la siguiente sucesión \(x_{2n}=n,x_{2n+1}=0\), para infinitas considere \(x_{2n}=n,x_{2n+1}=\frac1n\).
        \end{enumerate}
    \end{sol}
\end{ans}


\begin{prob}
    Sea \(x_n\) una sucesión de números enteros, demuestre que \(x_n\) siempre cumple al menos una de las siguientes propiedades:
    \begin{enumerate}[label=(\alph*)]
        \item Tiene una cantidad finita de términos distintos, en otras palabras el conjunto \(S=\{x_n:n\in\set{N}\}\) es finito.
        \item Es no acotada.
    \end{enumerate}
\end{prob}

\begin{ans}
    \begin{sol}
        Sea \(S=\{\abs{x_n}:n\in\set{N}\}\subset\set{N}\), se ve que si \(x_n\) es acotada existe un \(M\in\set{R}\) tal que \(\forall a\in S a<M\), ahora, por propiedad arquimediana existe un \(n\in\set{N}\) tal que \(n>M\), por lo que se tiene que \(S\subseteq\{0,1,\ldots,n\}\), por lo que \(\abs{S}\leq n+1\), por lo que \(S\) tiene finitos elementos. Ahora si, \(x_n\) no tiene finitos términos se tiene que \(\abs{S}=\infty\), pero si \(x_n\) es acotado se tiene que \(S\) es finito, por lo que \(x_n\) no es acotada.
    \end{sol}
\end{ans}

\end{document}