\documentclass{ayudantia}
\usepackage{multicol}
\title{Ayudantía 13}
\date{2020-10-13}
\course{MAT1106 --- Introducción al Cálculo}

% Comment for final compile
\ifx\condition\undefined
\def\condition{1}
\fi

\ifx\condition\undefined
\immediate\write18{ pdflatex -synctex=1 -output-directory="../Enunciados" --jobname="Enunciado\jobname" "\gdef\string\condition{0} \string\input\space\jobname"} 
\immediate\write18{ pdflatex -synctex=1 -output-directory="../Soluciones" --jobname="Solucion\jobname" "\gdef\string\condition{1} \string\input\space\jobname"} 

\immediate\write18{ cd "../Enunciados" && rm *.aux *.log *.out}
\immediate\write18{ cd "../Soluciones" && rm *.aux *.log *.out}

\expandafter\stop
\fi

\ifcase\condition
\excludecomment{ans}
\or
\includecomment{ans}
\fi

\begin{document}
\maketitle


\begin{prob}
    Demuestre que las siguientes sucesiones convergen a cero:
    \begin{enumerate}
        \item \(\ds x_n=\frac{1+2+\ldots+n}{n^3}\)
        \item \(\ds x_n=\frac{1+3+\ldots+(2n-1)}{n^3}\)
        \item \(\ds x_n=\frac{1+4+\ldots+n^2}{n^4}\)
    \end{enumerate}
\end{prob}

\begin{ans}
    \begin{sol}

    \end{sol}
\end{ans}



\begin{prob}
    Sea \(x_n\) una sucesión. Demuestre que \(\lim_{n\rightarrow\infty}x_n=0\) si y solo si para todo \(k\in\set{R}\) \(\lim_{n\rightarrow\infty}k\cdot x_n=0\).
\end{prob}

\begin{ans}
    \begin{sol}

    \end{sol}
\end{ans}



\begin{prob}
    Sea \(x_n\) tal que \(\lim_{n\rightarrow\infty}x_n=0\), y sea \(y_n\) una sucesión acotada, demuestre que \(\lim_{n\rightarrow\infty}x_ny_n=0\).
\end{prob}

\begin{ans}
    \begin{sol}

    \end{sol}
\end{ans}



\begin{prob}
    Sea \(x_n\) tal que \(x_n\neq0\) para todo \(n\in\set{N}\). Demuestre que si \(\lim_{n\rightarrow\infty}x_n=0\), entonces \(\frac1{x_n}\) no está acotada.
\end{prob}

\begin{ans}
    \begin{sol}

    \end{sol}
\end{ans}

\end{document}


