\documentclass{ayudantia}
\usepackage{multicol}
\title{Ayudantía 13}
\date{2020-10-13}
\course{MAT1106 --- Introducción al Cálculo}

% Comment for final compile
%\ifx\condition\undefined
%\def\condition{1}
%\fi

\ifx\condition\undefined
\immediate\write18{ pdflatex -synctex=1 -output-directory="../Enunciados" --jobname="Enunciado\jobname" "\gdef\string\condition{0} \string\input\space\jobname"} 
\immediate\write18{ pdflatex -synctex=1 -output-directory="../Soluciones" --jobname="Solucion\jobname" "\gdef\string\condition{1} \string\input\space\jobname"} 

\immediate\write18{ cd "../Enunciados" && rm *.aux *.log *.out}
\immediate\write18{ cd "../Soluciones" && rm *.aux *.log *.out}

\expandafter\stop
\fi

\ifcase\condition
\excludecomment{ans}
\or
\includecomment{ans}
\fi

\begin{document}
\maketitle


\begin{prob}
    Demuestre que las siguientes sucesiones convergen a cero:
    \begin{enumerate}
        \item \(\ds x_n=\frac{1+2+\ldots+n}{n^3}\)
        \item \(\ds x_n=\frac{1+3+\ldots+(2n-1)}{n^3}\)
        \item \(\ds x_n=\frac{1+4+\ldots+n^2}{n^4}\)
    \end{enumerate}
\end{prob}

\begin{ans}
    \begin{sol}
        \begin{enumerate}
            \item Notemos que \(x_n=\frac1{n^3}\sum_{k=1}^nk\), luego veamos que
            \begin{align*}
                \abs{x_n}&=\frac1{n^3}\sum_{k=1}^{n}k\\
                &\leq \frac1{n^3}\sum_{k=1}^{n}n\\
                &\leq \frac1{n^3}\cdot n^2\\
                &\leq \frac1n\\
            \end{align*}
            Dado \(\varepsilon>0\) sea \(n_0=\floor{\frac1\varepsilon}+1\), se nota que para \(n\geq n_0\) se tiene que \(\frac1n<\varepsilon\), por lo que por transitividad se tiene que \(\abs{x_n}<\varepsilon\). Lo que nos dice que \(\lim_{n\rightarrow\infty}x_n=0\).
            \item Notemos que \(x_n=\frac1{n^3}\sum_{k=1}^n2k-1\), luego veamos que
            \begin{align*}
                \abs{x_n}&=\frac1{n^3}\sum_{k=1}^n2k-1\\
                &\leq\frac1{n^3}\sum_{k=1}^n2n-1\\
                &\leq\frac1{n^3}(2n^2-n)\\
                &\leq\frac2n-\frac1{n^2}\\
                &\leq\frac2n+\frac1{n^2}\\
                &\leq\frac2n+\frac1n\\
                &\leq\frac3n\\
            \end{align*}
            Dado \(\varepsilon>0\) sea \(n_0=\floor{\frac3\varepsilon}+1\), se nota que para \(n\geq n_0\) se tiene que \(\frac3n<\varepsilon\), por lo que por transitividad se tieneque \(\abs{x_n}<\varepsilon\). Lo que nos dice que \(\lim_{n\rightarrow\infty}x_n=0\).
            \item Notemos que \(x_n=\frac1{n^4}\sum_{k=1}^nk^2\), luego veamos que
            \begin{align*}
                \abs{x_n}&=\frac1{n^3}\sum_{k=1}^nk^2\\
                &\leq\frac1{n^4}\sum_{k=1}^nn^2\\
                &\leq\frac1{n^4}\sum_{k=1}^nn^2\\
                &\leq\frac1{n^4}\cdot n^3\\
                &\leq\frac1n\\
            \end{align*}
            Usando el mismo argumento que se usa para 1), se tiene que \(\lim_{n\rightarrow\infty}x_n=0\).
        \end{enumerate}
    \end{sol}
\end{ans}



\begin{prob}
    Sea \(x_n\) una sucesión. Demuestre que \(\lim_{n\rightarrow\infty}x_n=0\) si y solo si para todo \(k\in\set{R}\) \(\lim_{n\rightarrow\infty}k\cdot x_n=0\).
\end{prob}

\begin{ans}
    \begin{sol}
        Se nota que \(\impliedby\) es trivial tomando \(k=1\).

        Para \(\implies\), si \(k=0\), se tiene trivialmente, por lo que para \(k\neq0\) sea \(\varepsilon>0\) se tiene que existe un \(n_0\in\set{N}\) tal que para \(n\geq n_0\) se tiene que \(\abs{x_n}<\frac\varepsilon{\abs{k}}\) lo que es equivalente a \(\abs{k\cdot x_n}<\varepsilon\), por lo que \(\lim_{n\rightarrow\infty}k\cdot x_n=0\).
    \end{sol}
\end{ans}



\begin{prob}
    Sea \(x_n\) tal que \(\lim_{n\rightarrow\infty}x_n=0\), y sea \(y_n\) una sucesión acotada, demuestre que \(\lim_{n\rightarrow\infty}x_ny_n=0\).
\end{prob}

\begin{ans}
    \begin{sol}
        Por definición de acotado de tiene que \(\exists M\in\set{R}\) tal que \(\forall n\in\set{N}\) \(0\leq\abs{y_n}<M\), por lo que dado un \(\varepsilon>0\) existe un \(n_0\in\set{N}\) tal que \(n\geq n_0\) da que \(\abs{x_n}<\frac\varepsilon{M}\), ahora se ve que \(\abs{y_nx_n}<M\abs{x_n}<M\cdot\frac\varepsilon{M}=\varepsilon\), por lo que se tiene que \(\lim_{n\rightarrow\infty}x_ny_n=0\).
    \end{sol}
\end{ans}



\begin{prob}
    Sea \(x_n\) tal que \(x_n\neq0\) para todo \(n\in\set{N}\). Demuestre que si \(\lim_{n\rightarrow\infty}x_n=0\), entonces \(\frac1{x_n}\) no está acotada.
\end{prob}

\begin{ans}
    \begin{sol}
        Usando contrapositiva, si \(\frac1{x_n}\) es acotada, se tiene que existe un \(M\in\set{R}\) tal que para todo \(n\in\set{N}\) se tiene \(\abs{\frac1{x_n}}<M\), esto se puede reescribir como \(\frac1{M}<\abs{x_n}\), por lo que se tiene que \(x_n\) no puede converger a \(0\).
    \end{sol}
\end{ans}

\end{document}


