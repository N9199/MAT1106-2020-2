\documentclass{ayudantia}
\usepackage{multicol}
\title{Ayudantía 21}
\date{2020-11-12}
\course{MAT1106 --- Introducción al Cálculo}

% Comment for final compile
%\ifx\condition\undefined
%\def\condition{1}
%\fi

\ifx\condition\undefined
\immediate\write18{ pdflatex -synctex=1 -output-directory="../Enunciados" --jobname="Enunciado\jobname" "\gdef\string\condition{0} \string\input\space\jobname"} 
\immediate\write18{ pdflatex -synctex=1 -output-directory="../Soluciones" --jobname="Solucion\jobname" "\gdef\string\condition{1} \string\input\space\jobname"} 

\immediate\write18{ cd "../Enunciados" && rm *.aux *.log *.out}
\immediate\write18{ cd "../Soluciones" && rm *.aux *.log *.out}

\expandafter\stop
\fi

\ifcase\condition
\excludecomment{ans}
\or
\includecomment{ans}
\fi

\begin{document}
\maketitle

\begin{prob}
    Sea \(x_n\) una sucesión de Cauchy. Demuestre que \(x_n\) converge.
\end{prob}

\begin{ans}
    \begin{sol}

    \end{sol}
\end{ans}



\begin{prob}
    Sean \(x_n\) e \(y_n\) sucesiones de Cauchy. Sea \(z_n=x_n+y_n\), demuestre que es de Cauchy.
\end{prob}

\begin{ans}
    \begin{sol}

    \end{sol}
\end{ans}



\begin{prob}
    Sea \(E\subseteq\set{R}\), demuestre que \(E\) es denso si y solo si para todo \(x\in\set{R}\) y \(\varepsilon>0\) existe un \(e\in E\) tal que \(\abs{x-e}<\varepsilon\).
\end{prob}

\begin{ans}
    \begin{sol}

    \end{sol}
\end{ans}
\end{document}









