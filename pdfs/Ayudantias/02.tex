\documentclass{ayudantia}

\title{Ayudantía 01}
\date{2020-08-19}
\course{MAT1106 --- Introducción al Cálculo}

% Comment for final compile
\ifx\condition\undefined
\def\condition{1}
\fi

\ifx\condition\undefined
\immediate\write18{ pdflatex -synctex=1 -output-directory="../Enunciados" --jobname="Enunciado\jobname" "\gdef\string\condition{0} \string\input\space\jobname"} 
\immediate\write18{ pdflatex -synctex=1 -output-directory="../Soluciones" --jobname="Solucion\jobname" "\gdef\string\condition{1} \string\input\space\jobname"} 

\immediate\write18{ cd "../Enunciados" && rm *.aux *.log *.out}
\immediate\write18{ cd "../Soluciones" && rm *.aux *.log *.out}

\expandafter\stop
\fi

\ifcase\condition
\excludecomment{ans}
\or
\includecomment{ans}
\fi

\begin{document}
\maketitle

\begin{prob}
    Sean \(a,b\) tales que \(ab=1\). Demuestre que \(a^2+b^2\geq2\).
\end{prob}

\begin{ans}
    \begin{sol}
        Se recuerda que \((a-b)^2\geq0\), por lo que \((a^2+b^2)-2ab\geq0\), como \(ab=1\) se tiene que \((a^2+b^2)-2\geq0\) por lo que usando la definición de \(\geq\) se tiene que \(a^2+b^2\geq2\).
    \end{sol}
\end{ans}


\begin{prob}
    Sean \(a,b\in\set{R}\)
    \begin{enumerate}
        \item Demuestre que \(a^2+ab+b^2\geq0\), y determine cuando se cumple la igualdad.
        \item Demuestre que \(a^3-b^3=(a-b)(a^2+ab+b^2)\).
        \item Concluya que \(a^3>b^3\) si y solo si \(a>b\).
    \end{enumerate}\
\end{prob}

\begin{ans}
    \begin{sol}
        \begin{enumerate}
            \item Se nota que \(a^2+ab+b^2=\frac{a^2+b^2}2+\frac{(a+b)^2}2\), por lo que \(a^2+ab+b^2\geq0\). Dado eso, se tiene que \(a^2+ab+b^2=0\) si y solo si \(a^2+b^2=0\) por lo que \(a=b=0\).
            \item Se ve lo siguiente:
            \begin{align*}
                (a-b)(a^2+ab+b^2)&=a(a^2+ab+b^2)-b(a^2+ab+b^2)\\
                &=a^3+a^2b+ab^2-ba^2-ab^2-b^3\\
                &=a^3-b^3\\
            \end{align*}
            \item Con lo anterior, se nota que si \(a^3>b^3\) o \(a>b\), se tiene que \(a\neq b\), por lo que \(a^2+ab+b^2>0\). Por ende
            \begin{align*}
                a^3>b^3&\iff a^3-b^3>0\\
                &\iff (a-b)(a^2+ab+b^2)>0\\
                &\iff a-b>0\cdot(a^2+ab+b^2)^{-1}\\
                &\iff a-b>0\\
                &\iff a>b\\
            \end{align*}
        \end{enumerate}
    \end{sol}
\end{ans}


\begin{prob}
    Sean \(a,b,c,d\in\set{R}\) tales que \(a<b\) y \(c<d\). Pruebe que \(ad+bc<ac+bd\)
\end{prob}

\begin{ans}
    \begin{sol}
        Notar que \(0<b-a\) y \(0<d-c\), por lo que
        \begin{align*}
            0<(b-a)(d-c)&\iff 0<b(d-c)-a(d-c)\\
            &\iff 0<bd-bc-ad+ac\\
            &\iff ad+bc<ac+bd
        \end{align*}
        Con lo que se tiene lo pedido.
    \end{sol}
\end{ans}


\begin{prob}
    Demuestre que si \(L-\varepsilon\leq M\) para todo \(\varepsilon>0\), entonces \(L\leq M\)
\end{prob}

\begin{ans}
    \begin{sol}
        Por contradicción, se asume que \(L>M\), luego sea \(\varepsilon=\frac{L-M}2>0\), entonces
        \begin{align*}
            L-\varepsilon\leq M&\iff L-\frac{L-M}2\leq M\\
            &\iff 2L-(L-M)\leq 2M\\
            &\iff 2L-L+M\leq 2M\\
            &\iff L+M\leq 2M\\
            &\iff L\leq M\\
        \end{align*}
        Lo último es una contradicción, por lo que \(L\leq M\).
    \end{sol}
\end{ans}


\begin{prob}
    Se define el mínimo entre \(a\) y \(b\) como
    \begin{equation*}
        \min(a,b)=\begin{cases}
            a&\text{si }a\leq b\\
            b&\text{si }a>b\\
        \end{cases}
    \end{equation*}
    Demuestre que \(\abs{x}=-\min(x,-x)\).
\end{prob}

\begin{ans}
    \begin{sol}
        Por casos, si \(x\geq0\) se tiene que \(\abs{x}=x\), ahora como \(x\geq0\) se tiene que \(-x\leq 0\leq x\), por lo que \(\min(x,-x)=-x\), más aún \(-\min(x,-x)=x=\abs{x}\). Si \(x<0\), se tiene que \(\abs{x}=-x\) y \(-x>0>x\), por lo tanto \(\min(x,-x)=x\), y \(-\min(x,-x)=-x=\abs{x}\).
    \end{sol}
\end{ans}


\begin{prob}
    Se define el máximo entre \(a\) y \(b\) como
    \begin{equation*}
        \max(a,b)=\begin{cases}
            a&\text{si }a\geq b\\
            b&\text{si }a<b\\
        \end{cases}
    \end{equation*}
    Demuestre que \(\max(a,b)=\frac{a+b}2+\frac{\abs{a-b}}2\).
\end{prob}

\begin{ans}
    \begin{sol}
        De nuevo, por casos, si \(a\geq b\) se tiene que \(\abs{a-b}=a-b\), por lo que
        \begin{align*}
            \frac{a+b}2+\frac{\abs{a-b}}2&=\frac{a+b}2+\frac{a-b}2\\
            &=\frac{a+b+a-b}2\\
            &=\frac{2a}2\\
            &=a\\
            &=\max(a,b)\\
        \end{align*}
        Si \(a<b\) se tiene que \(abs{a-b}=b-a\), entonces
        \begin{align*}
            \frac{a+b}2+\frac{\abs{a-b}}2&=\frac{a+b}2+\frac{b-a}2\\
            &=\frac{a+b+b-a}2\\
            &=\frac{2b}2\\
            &=b\\
            &=\max(a,b)\\
        \end{align*}
        Con lo que se tiene lo pedido.
    \end{sol}
\end{ans}


\end{document}