\documentclass{ayudantia}
\usepackage{multicol}
\title{Ayudantía 14}
\date{2020-10-15}
\course{MAT1106 --- Introducción al Cálculo}

% Comment for final compile
%\ifx\condition\undefined
%\def\condition{1}
%\fi

\ifx\condition\undefined
\immediate\write18{ pdflatex -synctex=1 -output-directory="../Enunciados" --jobname="Enunciado\jobname" "\gdef\string\condition{0} \string\input\space\jobname"} 
\immediate\write18{ pdflatex -synctex=1 -output-directory="../Soluciones" --jobname="Solucion\jobname" "\gdef\string\condition{1} \string\input\space\jobname"} 

\immediate\write18{ cd "../Enunciados" && rm *.aux *.log *.out}
\immediate\write18{ cd "../Soluciones" && rm *.aux *.log *.out}

\expandafter\stop
\fi

\ifcase\condition
\excludecomment{ans}
\or
\includecomment{ans}
\fi

\begin{document}
\maketitle


\begin{prob}
    Sea \(x_n=1-0,\underbrace{99\ldots9}_{\text{n 9s}}\). Demuestre que \(\lim_{n\rightarrow\infty}x_n=0\)
\end{prob}

\begin{ans}
    \begin{sol}

\end{sol}
\end{ans}



\begin{prob}
    Demuestre que las siguientes sucesiones convergen a cero:
    \begin{enumerate}
        \item \(x_n=\frac{\sin(n)}n\)
        \item \(x_n=n\sin(1/n)-1\)
    \end{enumerate}
\end{prob}

\begin{ans}
    \begin{sol}
        \begin{enumerate}
            \item \(\abs{\sin(n)}\leq1\)
            \item Ver circulo unitario y notar que \(\sin(\theta)\leq\theta\leq\tan(\theta)\), para \(\theta\leq\tan\theta\) ver áreas del triángulo exterior
        \end{enumerate}
\end{sol}
\end{ans}



\begin{prob}
    Demuestre que si \(\lim_{n\rightarrow\infty}x_n=\infty\), entonces \(\lim_{n\rightarrow\infty}\frac1{x_n}=0\). ¿Es verdad el recíproco? Si lo es, demuestrelo, si no lo es, encuentre condiciones necesarias y suficientes.
\end{prob}

\begin{ans}
    \begin{sol}

    \end{sol}
\end{ans}




\begin{prob}
    Sea \(x_n\) una sucesión de enteros que converge a \(L\), demuestre que \(x_n\) es eventualmente constante.
\end{prob}

\begin{ans}
    \begin{sol}

\end{sol}
\end{ans}


\begin{prob}
    Demuestre que \(\lim_{n\rightarrow\infty}x_n=L\) si y solo si \(\lim_{n\rightarrow\infty}y_n=0\), donde \(y_n=x_n-L\).
\end{prob}

\begin{ans}
    \begin{sol}

\end{sol}
\end{ans}

\end{document}



