\documentclass{ayudantia}
\usepackage{multicol}
\title{Ayudantía 14}
\date{2020-10-15}
\course{MAT1106 --- Introducción al Cálculo}

% Comment for final compile
\ifx\condition\undefined
\def\condition{1}
\fi

\ifx\condition\undefined
\immediate\write18{ pdflatex -synctex=1 -output-directory="../Enunciados" --jobname="Enunciado\jobname" "\gdef\string\condition{0} \string\input\space\jobname"} 
\immediate\write18{ pdflatex -synctex=1 -output-directory="../Soluciones" --jobname="Solucion\jobname" "\gdef\string\condition{1} \string\input\space\jobname"} 

\immediate\write18{ cd "../Enunciados" && rm *.aux *.log *.out}
\immediate\write18{ cd "../Soluciones" && rm *.aux *.log *.out}

\expandafter\stop
\fi

\ifcase\condition
\excludecomment{ans}
\or
\includecomment{ans}
\fi

\begin{document}
\maketitle


\begin{prob}
    Sea \(x_n=1-0,\underbrace{99\ldots9}_{\text{n 9s}}\). Demuestre que \(\lim_{n\rightarrow\infty}x_n=0\)
\end{prob}

\begin{ans}
    \begin{sol}
        Se nota que \(x_n=1-\sum_{k=1}^n\frac9{10^k}\), y usando inducción simple se puede demostrar que \(x_n=\frac1{10^{n+1}}\), usando la desigualdad de Bernoulli se tiene que \(10^n\geq1+9n>n\) por lo que se tiene que \(\frac1n>\frac1{10^n}\), por lo que para un \(\varepsilon>0\) se toma \(n_0=\floor{\frac1\varepsilon}+2\) y se tiene lo pedido.
    \end{sol}
\end{ans}



\begin{prob}
    Demuestre que las siguientes sucesiones convergen a cero:
    \begin{enumerate}
        \item \(x_n=\frac{\sin(n)}n\)
        \item \(x_n=n\sin(1/n)-1\)
    \end{enumerate}
\end{prob}

\begin{ans}
    \begin{sol}
        \begin{enumerate}
            \item Como \(\abs{\sin(n)}\leq1\) se tiene que \(\abs{x_n}\leq\frac1n\), por lo que tomando \(n_0=\floor{\frac1\varepsilon}+1\) se tiene que \(\lim_{n\rightarrow\infty}x_n=0\).
            \item Ver circulo unitario y notar que \(\sin(\theta)\leq\theta\leq\tan(\theta)\), para \(\theta\leq\tan\theta\) ver áreas del triángulo exterior
        \end{enumerate}
    \end{sol}
\end{ans}



\begin{prob}
    Demuestre que si \(\lim_{n\rightarrow\infty}x_n=\infty\), entonces \(\lim_{n\rightarrow\infty}\frac1{x_n}=0\). ¿Es verdad el recíproco? Si lo es, demuestrelo, si no lo es, encuentre condiciones necesarias y suficientes.
\end{prob}

\begin{ans}
    \begin{sol}

    \end{sol}
\end{ans}




\begin{prob}
    Sea \(x_n\) una sucesión de enteros que converge a \(L\), demuestre que \(x_n\) es eventualmente constante.
\end{prob}

\begin{ans}
    \begin{sol}
        Sea \(\varepsilon>0\) se tiene que \(\frac1\varepsilon>0\), por lo que tomando \(R=\frac1\varepsilon\)
    \end{sol}
\end{ans}


\begin{prob}
    Demuestre que \(\lim_{n\rightarrow\infty}x_n=L\) si y solo si \(\lim_{n\rightarrow\infty}y_n=0\), donde \(y_n=x_n-L\).
\end{prob}

\begin{ans}
    \begin{sol}

    \end{sol}
\end{ans}

\end{document}



