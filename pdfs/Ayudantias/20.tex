\documentclass{ayudantia}
\usepackage{multicol}
\title{Ayudantía 18}
\date{2020-11-03}
\course{MAT1106 --- Introducción al Cálculo}

% Comment for final compile
%\ifx\condition\undefined
%\def\condition{1}
%\fi

\ifx\condition\undefined
\immediate\write18{ pdflatex -synctex=1 -output-directory="../Enunciados" --jobname="Enunciado\jobname" "\gdef\string\condition{0} \string\input\space\jobname"} 
\immediate\write18{ pdflatex -synctex=1 -output-directory="../Soluciones" --jobname="Solucion\jobname" "\gdef\string\condition{1} \string\input\space\jobname"} 

\immediate\write18{ cd "../Enunciados" && rm *.aux *.log *.out}
\immediate\write18{ cd "../Soluciones" && rm *.aux *.log *.out}

\expandafter\stop
\fi

\ifcase\condition
\excludecomment{ans}
\or
\includecomment{ans}
\fi

\begin{document}
\maketitle

\begin{prob}
    Demuestre que \(0,10100100010000100000\ldots\) es irracional.
\end{prob}

\begin{ans}
    \begin{sol}

    \end{sol}
\end{ans}



\begin{prob}
    Sean \(a=0,a_1a_2a_3\ldots\) y \(b=0,b_1b_2b_3\ldots\) reales con expansión decimal periódica de periodos \(k_1\) y \(k_2\) respectivamente, y además \(a_n+b_n\leq 9\) para todo \(n\in\set{N}\). Demuestre que \(a+b\) también es periódica, y encuentre su periodo.
\end{prob}

\begin{ans}
    \begin{sol}

    \end{sol}
\end{ans}



\begin{prob}
    Demuestre que si \(k=\frac{p}q\) (con \(p,q\in\set{N}\)) su expansión decimal tiene periodo a lo más \(k\).
\end{prob}

\begin{ans}
    \begin{sol}

    \end{sol}
\end{ans}



\begin{prob}
    ¿ Cuántas expansiones decimales distintas puede tener un número real?
\end{prob}

\begin{ans}
    \begin{sol}

    \end{sol}
\end{ans}
\end{document}








