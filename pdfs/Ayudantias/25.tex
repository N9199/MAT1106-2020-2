\documentclass{ayudantia}
\usepackage{multicol}
\title{Ayudantía 25}
\date{2020-12-03}
\course{MAT1106 --- Introducción al Cálculo}

% Comment for final compile
%\ifx\condition\undefined
%\def\condition{1}
%\fi

\ifx\condition\undefined
\immediate\write18{ pdflatex -synctex=1 -output-directory="../Enunciados" --jobname="Enunciado\jobname" "\gdef\string\condition{0} \string\input\space\jobname"} 
\immediate\write18{ pdflatex -synctex=1 -output-directory="../Soluciones" --jobname="Solucion\jobname" "\gdef\string\condition{1} \string\input\space\jobname"} 

\immediate\write18{ cd "../Enunciados" && rm *.aux *.log *.out}
\immediate\write18{ cd "../Soluciones" && rm *.aux *.log *.out}

\expandafter\stop
\fi

\ifcase\condition
\excludecomment{ans}
\or
\includecomment{ans}
\fi

\begin{document}
\maketitle

\begin{prob}
    Dado una función \(f:A\subseteq\set{R}\rightarrow \set{R}\) y un \(c\in A\) demuestre que las siguientes definiciones son equivalentes:
    \begin{enumerate}
        \item Para todo abierto \(N(f(c))\subseteq \set{R}\) que contiene a \(f(c)\) existe un abierto \(N(c)\subseteq A\) que contiene a \(c\) tal que para todo \(x\in N(C)\) se tiene que \(f(x)\in N(f(c))\).
        \item Para toda sucesión \(\{x_n\}_{n\in\set{N}}\subseteq A\) que converge a \(c\) se tiene que \(\lim_{n\rightarrow\infty}f(x_n)=f(c)\).
        \item Para todo \(\varepsilon>0\) existe un \(\delta>0\) tal que para todo \(x\in D\) \(\abs{x-c}<\delta\) implica que \(\abs{f(x)-f(c)}<\varepsilon\).
    \end{enumerate} 
\end{prob}

\begin{ans}
    \begin{sol}

    \end{sol}
\end{ans}



\begin{prob}
    Dado una función \(f:A\subseteq\set{R}\rightarrow \set{R}\) demuestre que las siguientes definiciones son equivalentes:
    \begin{enumerate}
        \item Para todo abierto \(V\subseteq\set{R}\) se tiene que \(f^{-1}(V)\) es abierto.
        \item Para todo cerrado \(V\subseteq\set{R}\) se tiene que \(f^{-1}(V)\) es cerrado.
        \item Para toda sucesión convergente \(\{x_n\}_{n\in\set{N}}\subseteq A\) se tiene que \(\lim_{n\rightarrow\infty}f(x_n)=f(\lim_{n\rightarrow\infty}x_n)\).
    \end{enumerate}
\end{prob}

\begin{ans}
    \begin{sol}

    \end{sol}
\end{ans}


\end{document}

